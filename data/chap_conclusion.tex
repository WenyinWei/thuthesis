\chapter{总结与展望}

在本文研究过程中几乎完全重构了 ERGOS 程序,使得磁谱分析的工具变得尤其方便使用。时局变化突兀,本文写作时适逢哈工大等校被禁止使用 Matlab,重写 ERGOS 可以说是适逢其时。初步建立了磁扰动场的评估标准,可以一定程度上评判是否合适施加设计的磁谱,我们可以单一地从生成随机场的角度来考虑,也可以用对磁谱归一化,考虑了 NTV 之后的品质因子进行判断。

从目前的扰动场的模拟结果,我们提出了对低 n 线圈进行最大化生成边界随机场的估计手段,它符合我们关于 $\Delta\Phi_{UL}$ 与生成边界随机层趋势的论断、并且我们能够得到低 n 线圈在不同环向模数的工作模式下,生成边界随机场能力强弱的比较。低 $n$ 线圈与螺旋电流丝之间的协同以产生边界随机场的能力也在本文中进行了模拟并得到了阶段性结果。

当用对磁谱幅值归一化的品质因子进行优化时,螺旋电流丝的场本身具有较强的共振分量,其有着很强的激发磁拓扑边界随机层的能力,故而品质因子较高。高 m 线圈、低 n 线圈(RMP 线圈)远没有其作用强烈,为了使其磁通适配,需要适当地增加高 $m$ 线圈和低 $n$ 线圈的电流,使得扰动场的“大小”尽量匹配。且数值实验表明,螺旋电流丝的磁扰动效果已经足够好,其他线圈在其基础上的优化并不显著。品质因子很难有倍数的提升,这是因为我们的品质因子在微扰时是无关幅度的,而螺旋电流丝本身在外部磁场已经相当程度上紧密贴合着螺旋度对应的曲线。换句话说,其本身谱形已经很难有改进的空间。不过,通过模拟,我们可以避免坏的情况,以避免施加不合适的线圈参数导致削弱了螺旋电流丝本身生成随机场的效果。

第四章节我们通过磁力线追踪和扩散的方法一定程度上刻画了扰动场对热负荷分布的影响,这一部分我们以螺旋电流丝的扰动场作为例子(主导模数是 $n=1$),观察到了偏滤器原打击点发生了打击点分裂的现象,和实验的结果在一定程度上能够吻合,这一模拟对进一步的热负荷优化有启发性。

\textit{注意},扰动场和等离子体之间的相互作用是复杂的,在本文中我们仅讨论了真空条件下扰动场激发磁岛链的效果,实际上等离子体的反馈会抑制住该效果,弱的扰动场会被屏蔽掉。当扰动场强度超过一定阈值时,则转变为穿透状态,此时和真空场的作用效果一定程度上是类似的。进一步的对屏蔽和穿透之间的转化关系研究可以通过模拟有理面上的屏蔽电流来揭示,又或者是磁流体模拟。

\section{展望}
\begin{enumerate}
    \item 虽然现在通过一些优化方法得到了品质因子较大的磁谱,但是受限于运算速度和计算时间,RMP 线圈的调节自由度有限,还未能将扰动场磁谱和磁面螺旋度贴合地较好。磁谱脊线向磁面螺旋度对应的曲线靠拢的趋势很弱,需要各线圈有更高的自由度。如 RMP 线圈采用更精细的单个线圈的电流控制,可以先测试 RMP 自身优化会得到什么结果。
    \item 考虑等离子体反馈后有理面产生磁岛会需要一定阈值的共振量,这是因为有理面受到影响后会产生屏蔽电流以屏蔽较弱的扰动场,只有当扰动场较强时会被穿透,激发出磁岛链。未来通过边界随机场的磁流体模拟研究可以对它有更深刻的认识。
\end{enumerate}


% \subsection{有限体积法}
% 在有限体积法进行计算的过程中,我们所储存的变量值是偏微分返程中守恒量在网格中的平均值。与之类似但有些不同的是,在有限元法中,我们用试函数使得所计算得到的函数是函数空间中最优的函数。

% \subsubsection{双流体模型}
% 将等离子体视为离子和电子相互渗透的双流体来看待,分别视为服从麦克斯韦分布的等离子体,相比于单流体的模型更能反映出电子和离子的不同响应特性和各自的流体特征。

% 各种扰动场对 ELM 的发生起到了显著的控制作用,而在扰动场施加时等离子体边界浮现的随机场则对粒子和热输运均影响深远。这一部分的研究设计将模拟中加入**。从湍流输运的角度解释磁场边界拓扑对输运的影响可能有较好的效果。


% \subsection{有限元、有限体积法}
% 偏微分方程的求解问题构成了现代工程领域许多重要的设计工作,计算框架和数值理论在各种高性能计算处理器的基础上的组合计算成为了现代工业设计的重要设计及优化工作。

% 有限元法(Finite Element Method, FEM)在多物理场分析中很成功,一方面它非常通用,另一方面有限元可以对不同计算域内物理问题适合的算法进行组合,这对于多物理问题而言是一个关键优势。

% 尽管有限元可以自然地处理弯曲和不规则几何图形,但有限元背后的数学相对有限体积法(Finite Volume Method, FVM)更复杂一些。有限体积法中自然地对物理偏微分方程组中的守恒量进行在网格上进行积分,离散值表示的是单元内该守恒量的积分平均值。于是有限体积法的重点在于如何通过单元(cell)的积分平均值插值表示单元边界的守恒量流量,即流函数。

% % 最后,对于时域时间上的仿真,为了效率往往需要使用显式求解器。但是有限元在实施此类技术方面存在困难,因此建议不要使用它。

% \begin{itemize}
%     \item EMC3-EIRENE
%     \item \textit{FEniCS\footnote{\url{https://fenicsproject.org/}}} 是开源(LGPLv3)的偏微分方程计算框架。 FEniCS 中丰富的 Python-C++ 接口使得科学工作者可以迅速地将他们面对的科学模型转化为有限元程序逻辑。在这里我们选取 FEniCS 是因为其后端的 PETSc\footnote{\url{https://www.mcs.anl.gov/petsc/}} 在支持 OpenMP、OpenCL 和 CUDA,在针对 PDE 的硬件优化上几乎无出其右,可以在几乎在任何并行计算硬件平台上得到快速应用。\underline{考虑到毕设时间的有限并且可能将考虑非线性等离子体响应},具备高层接口的 FEniCS 是快速实现偏微分方程的手段\inlinecite{FEniCS_LangtangenLogg2017}。
%     \item \textit{SU2\footnote{\url{https://su2code.github.io}}} 工具箱是基于 C++ 偏微分方程的求解分析工具并可以在给定条件基础上进行设计优化。这套工具是为计算流体力学和空气动力学形状优化而设计的,但它也能够进行扩展来处理任意几何的控制方程,例如位势流,弹性问题,电流力学问题,化学反应流以及其他问题。
%     \item \textit{MFEM\footnote{\url{https://mfem.org/}}} 与 FEniCS 类似,MFEM 也支持对后端采用 PETSc 进行并行加速。其在电磁场领域有过一些研究,在本论文中被采用作为辅助验证工具。
%     \item \textit{\mdddc \footnote{\url{https://w3.pppl.gov/~nferraro/m3dc1.html}}} 由美国普林斯顿大学等离子体实验室开发,是一个聚变等离子体界影响深远的非线性双流体模拟计算工具。但由于中美关系恶化及其代码闭源问题,\mdddc 的数值高精度算法及各类成果在本论文中仅作为数值理论的参考。
% \end{itemize}


% 环向低速转动的低 n 扰动场已经在 J-TEXT 等实验中应用,FEniCS 实现 toy 级别的应用
    
% One day in 100+ lines.

% \begin{equation}
%   \nabla \times\vect{H} = \vect{j}+\frac{\partial \vect{D}}{\partial t} 
% \end{equation}

% Forward Euler
% \begin{equation}
%   \int \vect{B}^{n+1}\cdot\vect{B}^* dx =\int \vect{B}^n\cdot \vect{B}^* - \Delta t  \nabla\times\vect{E}^n\cdot\vect{B}^* dx
% \end{equation}

% TODO:
% \begin{enumerate}
%   \item 和 ERGOS 静磁学毕奥萨伐尔定律进行单一线圈精度对比。
%   \item 比置零更精确的边界条件(PML 或其他吸收层边界)引入。
%   \item Maxwell 方程高阶算法和混合元(电磁场错开)的引入。
% \end{enumerate}




% \begin{figure}[t]
% \centering
% \subcaptionbox{二维铁包层内外极向均匀分布的异向电流丝产生的磁场分布}{%
%     \includegraphics[width=0.45\columnwidth,keepaspectratio]{fenics/test/fenics_electrostatic_case.png}
% }\hfill
% \subcaptionbox{环形线圈产生的磁场分布示意图}{%
%     \includegraphics[width=0.45\columnwidth,keepaspectratio]{fenics/test/fenics_coil_B.png}
% }%
% \label{fig:highm-pos}
% \caption{基于 FEniCS 进行的模拟结果}
% \end{figure}

