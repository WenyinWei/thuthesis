\chapter*{Reading Note}
Terms marked with \emph{color} mean they are not yet verified.

The Vlasov equation is studied as a governing equation describing the collisionless plasma physics, 
in which the anistropic velocity distribution contributes a significant influence to the dynamics of the system. When coupled with Maxwell equations as the electromagnetism field governing rule, Vlasov is capable to decide the dynamic 
scenario for particle-field interaction, named Vlasov-Maxwell system (VM). Furthermore, under the assumption that the electrostatic force dominates the interaction, magnetic force is omitted and then comes Vlasov-Poisson problem (VP).

\begin{equation}\text { (VP) }\left\{\begin{array}{l}
    \partial_{t} f+a(v) \cdot \nabla_{x} f+\mu \nabla_{x} \phi \cdot \nabla_{v} f=0 \\
    \Delta \phi=\rho(f):=\int_{\mathbb{R}^{3}} f(t, x, v) d v
\end{array}\right.\end{equation}
where, $\mu \in\{+,-\}, a(v) \in\{v, \hat{v}\},$ and $\hat{v}:=v / \sqrt{1+|v|^{2}}$. The sign of $\mu$ indicates different physical scenario, "+" for the plasma physics case and "-" for the stellar dynamics case.

Phase space distribution $\ftxv\geq 0 (x\in \bbRx,v\in\bbRv,t\geq 0) $ with initial datum $f_{0}(\vx, \vv)=f(0, \vx, \vv)$ determines the particle density at $(t, \vx,\vv)$, \textit{i.e.}, the number of particles per unit volumn in phase space.

Some quantities have physical meaning are introduced as follows,
\begin{equation}\begin{array}{c}
    E(t, x)=\nabla_{x} \phi(t, x), \quad \phi=\frac{1}{4 \pi} \frac{1}{|x|} *\rho \\
    \quad \rho(t, x)=\int f(t, x, v) d v, \quad j(t, x)=\int a(v)f(t, x, v) d v
\end{array}\end{equation}

The more realistic equations consider relativistic effect when $a(v)=\hat{v}$ and limit the max velocity the particle can reach, transforming the Vlasov-Poisson to relativistic Vlasov-Poisson (RVP) and the Vlasov-Maxwell to relativistic Vlasov-Maxwell (RVM).
\begin{equation}\label{eq:rvm}\text { (VM) }\left\{\begin{array}{l}
f_{t}+a(v) \cdot \nabla_{x} f+\mu(E+a(v) \times B) \cdot \nabla_{v} f=0 \\
E_{t}=\nabla \times B-j \\
B_{t}=-\nabla \times E \\
\nabla \cdot E=\rho, \quad \nabla \cdot B=0
\end{array}\right.\end{equation}
\\
\\

For classical solutions, it is well known that the existence and
uniqueness result of the Vlasov-Poisson system solution have been presented by Iordanskii \footnote{The paper [16] listed in the reference of \cite{1991InMat.105..415L} is missing, refered as Iordanskii, S.V.: The Cauehy problem for the kinetic equation of plasma. Transl., II. Ser.,
Am. Math. Soc. 35, 351-363 (1964)} in dimension
1, \cite{ukai1978classical} in dimension 2, \cite{bardos1985global} in dimensions
3 for small data. 
The case of (nearly) symmetric data has been treated by \cite{batt1977global}, \cite{wollman1980global}, \cite{horst1981classical}, \cite{schaeffer1987global}. \cite{schaeffer1987global} treated the relativistic case of symmetric data in one paper.


\cite{glassey_symmetric_1985} stated the conservative characteristic of mass and energy, $\|\rho(t,\cdot)\|$ has a bound relying on $\|f_0\|_\infty$ and $\mathscr{E}_0$ in plasma case and proved the existence of global classical spherically symmertric solutions to the Cauchy problem with compact support for the 3D RVP system in the plasma physics case. It turns out that, for the case of stellar dynamics, the "small" data with $40M^{2/3}\|f_0\|_\infty^{1/3}$ will sustain the solution globally while the "large" data for which $-\infty <\mathscr{E}_0<0$, the solution blows up in finite time.

\cite{1991InMat.105..415L}, based on the representation formula built by the characteristic method considering the source term, proves the propagation of moments in $v$ higher than 3.
More precisely, if $|v|^m f_0\in L^1(\bbR^6)\text{ for all }m<m_0,\text{with } m_0> 3$, then we build a solution
of Vlasov-Poisson equations satisfying $|v|^m \ftxv\in L^1(\bbR^6)\text{ for all }m<m_0,\text{with } m_0> 3$ for any $t>0$. Moreover, for $m_0>6$, \emph{Sobolev injections}  deduces that $E\in L^\infty( [0, T]\times \bbRRR )$ for any $t>0$, and,
following Horst \footnote{The paper [14] listed in the reference of \cite{1991InMat.105..415L} is missing, refered as Horst, E.: Global strong solutions of Vlasov's Equation. Necessary and sufficient conditions
for their existence. Partial differential equations. Banach Cent. Publ. 19, 143 153 (1987)}, $f$ is a smooth function if $f_0(x, v)$ is smooth. This improves
a recent result of Pfaffelm6ser \footnote{The paper [19] listed in the reference of \cite{1991InMat.105..415L} is missing, refered as Pfaffelmoser Global classical (footnote from last page) solutions of the Vlasov-Poisson system in three dimensions
for general initial date. (Preprint)} who has shown independently of our work
the existence of a classical solution when )Co is, roughly speaking, Lipschitz
continuous in the space $L_v^{1}(L_x^\infty)$. 


The existence of global solution for the 3D RVM system with finite energy initial data has been investigated extensively by many authors. \cite{glassey1986singularity} deduced the result that the classical solution can be globally
extended with the strong assumption that $f$ has compact support in $v$ for all the time. \cite{staffilani2001fourier} presents a new perspective with Fourier analysis to the study of 3D RVM system and acquired the same result. 
% See also [1, 12, 28] for improvements of this result.

Continuation criterion for the global existence of the Vlasov-Maxwell system
may be a more relaxed condition than assuming the compact support in $v$ assumption. \cite{robert1989largevelocity} proved that, if the initial data decay at rate $|v|^{−7}$ as $|v| \to\infty$ and
the imposed assumption that bound “$\iint_{\bbRRR} |v|f(t, x, v)dv ≤ \text{Constant}$” holds, then the solution extends globall.
% Recently, an interesting new continuation criterion was given by \cite{2016ArRMA.219..445L}, which says that a regular solution can be
% extended as long as k(1+|v|2)θ/2f(t, x, v)kLq xL1 v remains bounded for θ > 2/q, 2 < q ≤ +∞. 
\cite{2016ArRMA.219..445L}, \cite{KUNZE20154413}, \cite{pallard2015criterion}
and \cite{PATEL20181841} have developed recent improvements on the continuation criterion. To release the limitation of the compact support and get rid of the assumptions on the solution itself, \cite{wang_propagation_2018} studied the propagation of regularity and the long time behavior of the 3D RVM system for suitably small initial data.
% Although our main interest is in 3D, we also refer
% readers to [13, 25, 26] for the corresponding results in other dimensions

\cite{chen_moments_2019} recently indicated that with the space torus topology $\bbT^3$, the weak solution of Vlasov-Poisson system exists globally and the velocity moment of order $>3$ can propagate, provided that the initial datum $0<f_{0} \in L^{1} \cap L^{\infty}\left(\mathbb{T}^{3} \times \mathbb{R}^{3}\right)$. 


\section*{Propagation of moments and regularity for the 3-dimensional Vlasov-Poisson system}


The most import result in \cite{1991InMat.105..415L} is the theorem of moment propagation,

\begin{theorem}
    Let $f_0\geq 0$,$f_0\in L^1\bigcap L^\infty (\bbRRR\times \bbRRR)$. We assume that 
    \begin{equation}
        \int_{\bbRRR \times \bbRRR}|\vv|^{m} f_{0}(\vx, \vv) d \vx d \vv<+\infty \quad \text { if } m<m_{0},
    \end{equation}
    where $3<m_0$. Then, there exists a solution $f\in C(\bbR^+; L^p(\bbRRR\times\bbRRR))\bigcap L^\infty (\bbR^+;L^\infty(\bbRRR\times \bbRRR))$ (for all $1\leq p < +\infty$) of Vlasov-Poisson system satisfying 
    \begin{equation}
    \sup _{t \in[0, T]} \iint_{ \mathbb{R}^{3} \times \mathbb{R}^{3}}|v|^{m} f(t, x, v) d x d v<+\infty
    \end{equation}
\end{theorem}

In the first step of the proof, we prove some general estimates
on E. Then, we conclude assuming that the time interval $(0, T)$ on which we
solve the Vlasov-Poisson system is small enough and that E is bounded in
$L^{3/2}(\bbRx)$, in fact which is false in general but holds true for bounded domains
of $\bbRx$. We relax this assumption on $E$ in a fourth and final step.

Due to the solution lying in general functional space $\mathscr{D}'$,  $
    E \cdot \nabla_{v} f=\operatorname{div}_{v}(E f)$ \footnote{There seems to be a typo in the paper where $E \cdot \nabla_{x} f=\operatorname{div}_{v}(E f)$}




\begin{proof}

(\romannum{1}) \textit{General estimates.} Simplify the proof by handling smooth solutions first, \textit{i.e.}, $C^\infty$ with compact support. The \emph{standard approximation method} (see[5,6]) would be enough to say the estimates are uniform and expand the function space, so don't worry. Treat the $E\cdot \nabla_v f$ on the RHS as a source term safely in Vlasov-Poisson equation \eqref{eq:vp}, trace the characteristic and then integrate in $v$ to acquire $\rho(t,x)$. 

\begin{equation}
\label{eq:rho_characteristic}
\rho(t, x)=-\operatorname{div}_{x} \int_{0}^{t} s \int_{\mathbf{R}^{3}}[E f(t-s, x-v s, v)] d v d s+\int_{\mathbf{R}^{3}} f_{0}(x-v t, v) d v\end{equation}


\begin{definition}
The supreme k-order moment in $|v|$ the solution has on the time interval $[0,t]$ is defined as follows:
\begin{equation}
    M_{k}(t)=\sup _{0 \leq s \leq t} \int|v|^{k} f(s) d x d v
\end{equation}
\end{definition}

The above equation \eqref{eq:rho_characteristic} is then \emph{transformed} to inequality concerning $E$
\begin{equation}\label{eq:E-norm-bound}
\|E(t,\cdot)\|_{m+3} \leq C+C\left\|\int_{0}^{t} s \int_{\mathbb{R}^{3}} E f(t-s, x-v s, v) d v d s\right\|_{m+3}\end{equation}
and \emph{deduces} the following inequality between the derivative of $M_{k}(t)$ and its power $\frac{k+2}{k+3} $.
\begin{equation}\label{eq:Mk-derivative-bound}\begin{aligned}
    \frac{d}{d t} M_{k}(t) & \leq | \frac{d}{d t} \int|v|^{k} f(t) d x d v | \\
    & \leq C\|E(t)\|_{3+k} M_{k}(t)^{\frac{k+2}{k+3}},
\end{aligned}\end{equation}
with which one can imply by Gronwall lemma that,
\begin{equation}M_{k}(t) \leqq C\left\{M_{k}(0)+\left(t \sup _{s \in(0, t)}\|E(s)\|_{3+k}\right)^{k+3}\right\}\end{equation}
confirming the bound of $M_{k}(t)$, \textit{i.e.} the supreme k-order moment in $|v|$ the solution has on the time interval $[0,t]$, exists and restricted by the supreme $E$ during the time $(0,t)$


(\romannum{2}) \textit{Small time estimates.} Based on the above $\|E(t,\cdot)\|_{m+3}$ estimate \eqref{eq:E-norm-bound} and \emph{its integration on the time interval} $(0,t_0)$ for any $r>3/2$ and $t_0\leq t\leq T$. 

\begin{equation}\label{eq:E-norm-restricted-by-M}\left\|\int_{0}^{t_{0}} s d s \sup _{\tau \in(0, T)}\left(\int_{R^{3}}|E|^{r}(\tau, y) \frac{d y}{s^{3}}\right)^{1 / r}\left(\int_{\mathbb{R}^{3}} f(t-s, x-v s, v) d v\right)^{1 / r^{\prime}}\right\|_{m+3}\|f\|_{\infty}^{\left(r^{\prime}-1\right) / r^{\prime}} \leq C t_{0}^{\gamma}\left(1+M_{m}(t)^{\delta}\right),\end{equation} where $r^\prime$ is the conjugate exponent of $r$, $1/r+1/r^\prime=1$, $1\leq r^\prime <3$ and $C$ only depends upon the initial data, $\gamma=2-3 / r>0$ and $\delta=3(k+1)/(m+3)^2>0$, $m_0>k>m$. Here comes the limitation for $m_0>3$.

On the other hand, $|E(t,\cdot)|$ is bounded by choosing $t_0=t$ in \eqref{eq:E-norm-bound},

$$\|E(t)\|_{m+3} \leqq C\left(1+t^{\gamma} M_{m}(t)\right)^{\delta}.$$

While in the first step, we introduced \eqref{eq:Mk-derivative-bound}, the derivative of  $M_{k}(t)$ is bounded by the product of $\|E(t,\cdot)\|_{3+k}$ and $M_{k}(t)^{\frac{k+2}{k+3}}$. For $k=m$, the above two equations jointly give

$$\frac{d}{d t} M_{m}(t) \leqq C\left(M_{m}(t)^{\frac{m+2}{m+3}}+t^{\gamma} M_{m}(t)^{\delta+\frac{m+2}{m+3}}\right) ,$$
by which we acquire a bound imposed by $M_{k}(t)$ itself on a small time interval $[0,t_0]$ exclusive of the appearance $\|E(t,\cdot)\|_{3+k}$.

In the following proof, step (\romannum{3}) firstly proves the boundedness of the $M_m(t)$ on any time interval $(0,T)$ in the case $E(t, \cdot)\in L^{3/2}$, and then step (\romannum{4}) expanded the range to the general case.

(\romannum{3}) \textit{The case when $E\in L^{3/2}$.} This is a stronger limitation on the functional space than the \textit{$E\in L^{3/2,\infty}(\bbRRR)$}\footnote{I don't really understand the definition of weak type lebesgue space}, relaxed by the author in the next step.

The estimate $\|E(t,\cdot)\|_{3+m}$ in \eqref{eq:E-norm-bound} combined with the bound acquired from step \romannum{2} \eqref{eq:E-norm-restricted-by-M} can give a bound of the derivative of $M_{m}(t)$ restricted by $M_{m}(t)$ itself on any time interval $(0,T)$.

\begin{equation}\frac{d}{d t} M_{m}(t) \leqq C\left(1+M_{m}(t)\left|\log M_{m}(t)\right|\right)\end{equation}
makes it possible to say $M_{m}(t)$ is bounded on any interval $(0,T)$.

(\romannum{4}) \textit{The general case.}
When $E\notin L^{3/2}(\bbRRR)$, similar to the step (\romannum{1}) treating $E\cdot \nabla_v f$ as a source term, the $E$ is decomposed into two parts: $E=\vect{E}_1+\vect{F}$

\begin{equation}
E_{1}=\frac{\alpha}{4 \pi}\left(\chi_{R}(x) \nabla \frac{1}{|x|}\right) * \rho,
\end{equation}where $0 \leq \chi_R\leq 1$ is smooth such that $\chi_R(x) = 1 $ if $|x|\leq R$ and $\chi_R(x)=0$ if $|x|\geq 2R$. But I am a little confused by the numerator why it is $\alpha$ \emph{rather than 1}.
\end{proof}

Treat the $\vE_1  \cdot \nabla_v f$ as source term on the RHS of the Vlasov-Poisson equation and trace the characteristic again like step (\romannum{1}), then it yields similar equation with \eqref{eq:E-norm-bound}.

\begin{equation}\|E(t)\|_{m+3} \leqq C+C\left\|\int_{0}^{t} \int_{\mathbb{R}^{3}}\left(\frac{\partial y}{\partial V}\right) E_{1} f(t-s, X(s), V(s)) d s d v\right\|_{m+3}\end{equation}

According to the \emph{standard approximation method}, it is said that it can keep the desired estimate, but I have not yet understood it.

\section*{On symmetric solutions of the relativistic Vlasov-Poisson system}






\cite{glassey_symmetric_1985} discusses about the existence problem of spherically symmertric solutions to the Cauchy problem for the 3D relativistic Vlasov-Poisson (RVP) system.


\begin{proposition}
    Let $f$ be a classical solution of (RVP) on some time interval $(0,T)$ with nonnegative data $f_0\in C_0^1(\bbR^6)$. Then the following properties hold:
    \begin{enumerate}
        \item If $f_0$ vanished for $|x|>k$, then $f(x,v,t)$ vanished for $|x|>t+k$ (casuality).
        \item The total mass is conserved, \textit{i.e.}, $\iint_{\bbR^6}fd\vect{v} d\vect{x} = constant \equiv M$.
        \item The total energy is conserved, \textit{i.e.},
        $$
\int_{\mathbb{R}^{3}}\left\langle\int_{R^{3}} \sqrt{1+|\vect{v}|^{2}} f d v+\frac{1}{2} \gamma|\vect{E}|^{2}\right\rangle d \vect{x}=\text { constant }=\mathscr{E}_{0}
$$
        \item If $\gamma=+1$, then there exists a constant $C$ (depending on $\|f_0\|_{\infty}$ and $\mathscr{E}_0$) such that $||\rho(t)||_{4/3}\leq C$ for $0 \leq t < T$.
    \end{enumerate}
\end{proposition}


The paper mainly talks about spherically symmertric solutions, \textit{i.e.}, the radial ones. Here is the simplified version of characteristic trace equation.

\begin{eqnarray}
    \frac{d R}{d s}&=&\frac{U \cos A}{\sqrt{1+U^{2}}} \\
    \frac{d U}{d s}&=&\gamma \frac{\cos A}{R^{2}} M(R, s) \\
    \frac{d A}{d s}&=&-\left(\gamma \frac{M(R, s)}{R^{2} U}+\frac{U}{R \sqrt{1+U^{2}}}\right) \sin A
\end{eqnarray}




\subsection{Velocity Bound in both cases}
\begin{lemma}
There exists a constant $C_{1}$ such that for $r \geq 0$ and $0 \leq t<T$
$$
|E(x, t)|=\frac{M(r, t)}{r^{2}} \leqq\left\{\begin{array}{ll}
{\min \left(M r^{-2}, 100 M^{1 / 3}\|\hat{f}\|_{\infty}^{2 / 3} P^{2}(t)\right)} & {\text { if } \gamma=-1} \\
{\min \left(M r^{-2}, C_{1} P^{5 / 3}(t)\right)} & {\text { if } \gamma=+1}
\end{array}\right.
$$
\end{lemma}

\begin{definition}
The highest speed the solution $f$ has on the time interval $[0,t]$.
$$
P(t)=\sup \{U(s, 0, r, u, \alpha): 0 \leq s \leq t,(r, u, \alpha) \in \text { support } f\}
$$
\end{definition}


\begin{lemma}
For all $z \geq 1$,
$$
\xi^{-1}(z) \leq\left[\left(z+A^{-1} B^{-1}\right)^{2}-1\right]^{1 / 2}
$$
\end{lemma}

\begin{lemma}
For all $t \in\left[0, T_{0}\right]$
$$
U(t) \leq U(0)+\xi^{-1}(\sqrt{\left.1+U^{2}(0)\right)}, \text { when } \gamma=-1
$$
\end{lemma}

\subsubsection{The stellar dynamics case}

\begin{theorem}
Let $f$ be a classical solution of (RVP) on some time interval $[0, T)$ with $\gamma=$
$-1$ and smooth, nonnegative, spherically symmetric data fwhich has compact support
and vanishes for $(r, u, \alpha) \notin(0, \infty) \times(0, \pi) .$ If $40 M^{2 / 3}\left\|f^{\circ}\right\|_{\infty}^{1 / 3}<1,$ then $P(t)$ is uniformly bounded on $[0, T)$, and hence (RVP) possesses a global classical solution.
\end{theorem}

\subsubsection{The plasma physics case}

\begin{definition}
    To control the derivative of characteristic $R(s)$, here comes a function: 
\begin{equation}
    G(r, t)=-\int_{r}^{\infty} \min \left(M \lambda^{-2}, C_{1} P^{5 / 3}(t)\right) d \lambda, r\geq 0 \text{ and } t \geq 0
\end{equation}

\end{definition}
\begin{lemma}
    \label{sqrt(1plusU)diff_leq_Gdiff}
    Assume either $\dot{R} \geqq 0$ on $\left[t_{1}, t_{2}\right]$ or $\dot{R} \leqq 0$ on $\left[t_{1}, t_{2}\right] .$ Assume  $ \gamma=+1$. Then

    \begin{equation}
|\sqrt{1+U^{2}\left(t_{2}\right)}-\sqrt{1+U^{2}\left(t_{1}\right)}| \leq\left|G\left(R\left(t_{2}\right), t_{2}\right)-G\left(R\left(t_{1}\right), t_{2}\right)\right|
    \end{equation}

\end{lemma}
\begin{remark}
    There exists a positive constant $C_2$
    \begin{equation}
        \label{Gdiff_leq_P0833}
        \left|G\left(r_{1}, t\right)-G\left(r_{2}, t\right)\right| \leqq C_{2} P^{5 / 6}(t) \text { for all } r_{1} \geqq 0, r_{2} \geqq 0, \text { and } t \geqq 0
        \end{equation}
\end{remark}


\begin{lemma}
Assume $\gamma=+1 . \dot{R}$ can be zero for at most one value of s. If $\dot{R}\left(t_{1}\right)=0,$ then
$R$ has an absolute minimum at $t_{1}$.
\end{lemma}

\begin{theorem}
Let $f$ be a classical solution of (RVP) on some time interval $[0, T)$ with $\gamma=+1$ and smooth, nonnegative, spherically symmetric data $f_0$ which has compact support and vanishes for $(r, u, \alpha) \notin(0, \infty) \times(0, \infty) \times(0, \pi) $. Then $P(t)$ is uniformly bounded on $[0, T)$, and hence (RVP) possesses a global classical solution.
\end{theorem}
\begin{proof}
    According to Lemma 1.6 and its following remark,
    \begin{equation}
        \label{sqrt(1plusU)diff_leq_Pdiff}
        \begin{aligned}
            \sqrt{1+U^{2}\left(t_{2}\right)} \leq& \sqrt{1+U^{2}\left(t_{1}\right)}+| G\left(R\left(t_{2}\right), t_{2}\right)-G\left(R\left(t_{1}\right), t_{2}\right)\\
            \leq& \sqrt{1+U^{2}\left(t_{1}\right)}+C_{2} P^{5 / 6}\left(t_{2}\right)
        \end{aligned}
    \end{equation}
    Either $\dot{R}$ never vanishes or vanishes at one value, 
    \begin{equation}
\sqrt{1+U^{2}(t)} \leq \sqrt{1+U^{2}(0)}+2 C_{2} P^{5 / 6}(t)
\end{equation}
holds for $t\in[0,T)$ as long as we apply Eq. \ref{sqrt(1plusU)diff_leq_Pdiff} at most twice. Naturally, 
\begin{equation} 
P(t) \leq \sqrt{1+P^{2}(t)} \leq \sqrt{1+P^{2}(0)}+2 C_{2} P^{5 / 6}(t)
\end{equation}
induces that $P(t)$ has upper bound.
\end{proof}

\subsection{Blow-up of Radial Solutions in the case of stellar dynamics system}

\begin{theorem}
Let $f_0$ be smooth, nonnegative, radial and of compact support on $\bbR^{6}$. Let $f(\vect{x}, \vect{v}, t)$ be a classical solution of (RVP) on an interval $0<t<T$ for which $-\infty<\mathscr{E}_{0}<0$. Then $T<\infty$.
\end{theorem}

\begin{proof}
    The "dilation identity" is used below, and here is its derivation.
    \begin{equation}
        \begin{aligned}
        \frac{d}{d t} \iint_{\mathbb{R}^{6}} \vect{x} \cdot \vect{v} f d \vect{v} d \vect{x}=& \iint_{\bbR^{6}} \vect{x} \cdot \vect{v}\left[-\hat{\vect{v}} \cdot \nabla_{x} f+\vect{E} \cdot \nabla_{v} f\right] d \vect{v} d \vect{x} = \cdots \text{(a lot omitted)}\\
        =& \iint_{R^{6}} \frac{|\vect{v}|^{2} f}{\sqrt{1+|\vect{v}|^{2}}} d \vect{v} d \vect{x}-\int_{R^{3}} \rho \vect{x} \cdot \vect{E} d \vect{x} = \cdots \text{(a lot omitted)}\\
        =& \iint_{R^{6}} \frac{|\vect{v}|^{2} f}{\sqrt{1+|\vect{v}|^{2}}} d \vect{v} d \vect{x} - \frac{1}{2} \int_{\bbR^{3}}|\nabla u|^{2} d \vect{x}\\
        =& \mathscr{E}_{0}-\iint_{\bbR^{6}} \frac{1}{\sqrt{1+|\vect{v}|^{2}}} f d \vect{v} d \vect{x}
        \end{aligned}
        \end{equation}
    
    Clearly, we have a coarse upper bound estimation of $\iint_{\mathbb{R}^{6}} \vect{x} \cdot \vect{v} f d \vect{v} d \vect{x}$, which is a part of $\frac{d}{d t} \iint_{\mathbb{R}^{6}} r^{2} \sqrt{1+|\vect{v}|^{2}} f d \vect{v} d \vect{x} $.
    $$\iint_{\mathbb{R}^{6}} \vect{x} \cdot \vect{v} f d \vect{v} d \vect{x}\leq \iint_{\mathbb{R}^{6}} \vect{x} \cdot \vect{v} f_0 d \vect{v} d \vect{x}+\mathcal{E}_0 t$$

    \begin{equation}
        \begin{aligned}
        \frac{d}{d t} \iint_{\mathbb{R}^{6}} r^{2} \sqrt{1+|\vect{v}|^{2}} f d \vect{v} d \vect{x} &=2 \iint_{\bbR^{6}} \vect{x} \cdot \vect{v} f d \vect{v} d \vect{x}-\int_{\bbR^{3}} r^{2} \vect{E} \cdot \vect{j} d \vect{x}\\
        ( \vect{j}=\int_{\mathbb{R}^{3}} \hat{\vect{v}} f d \vect{v}& \text{ and } \left|\int_{\mathbb{R}^{3}} r^{2} E \cdot \vect{j} d \vect{x}\right|   \leq M^{2})\\
        &\leq 2\left(\iint_{\mathbb{R}^{6}} \vect{x} \cdot \vect{v} f_0 d \vect{v} d \vect{x}+\mathscr{E}_{0} t\right)+M^{2}\\        
        &\leq \text{Constant} + 2\mathscr{E}_{0} t\\
        \Rightarrow 0\leq \iint_{\mathbb{R}^{6}} r^{2} \sqrt{1+|\vect{v}|^{2}} f d \vect{v} d \vect{x} &\leq C + Ct + \mathscr{E}_{0} t^2\\
    \end{aligned}
    \end{equation}

    However, since we assume $f_0$ is a "large" enough initial data, \textit{i.e.}, satisfying the hypothesis that $\mathscr{E}_0<0$. There must be some time the solution blows up.
\end{proof}


