\thusetup{
  %******************************
  % 注意:
  %   1. 配置里面不要出现空行
  %   2. 不需要的配置信息可以删除
  %******************************
  %
  %=====
  % 秘级
  %=====
  % secretlevel={秘密},
  % secretyear={10},
  %
  %=========
  % 中文信息
  %=========
  ctitle={三维扰动场对等离子体边界磁拓扑影响的协同优化模拟},
  cdegree={工学本科},
  cdepartment={工程物理系},
  cmajor={工程物理},
  cauthor={魏文崟},
  csupervisor={梁云峰教授},
  cassosupervisor={高喆教授}, % 副指导老师
  ccosupervisor={高喆教授}, % 联合指导老师
  % 日期自动使用当前时间,若需指定按如下方式修改:
  % cdate={超新星纪元},
  %
  % 博士后专有部分
  % catalognumber     = {},  % 可以留空
  % udc               = {},  % 可以留空
  % id                = {},  % 可以留空: id={},
  % cfirstdiscipline  = {},  % 流动站(一级学科)名称
  % cseconddiscipline = {},  % 专 业(二级学科)名称
  % postdoctordate    = {},  % 工作完成日期
  % postdocstartdate  = {},  % 研究工作起始时间
  % postdocenddate    = {},  % 研究工作期满时间
  %
  %=========
  % 英文信息
  %=========
  etitle={Collaborative Optimization of Multiple 3D Magnetic Perturbation Fields according to Their Effects on Plasma Edge Magnetic Topology in EAST tokamak},
  % 这块比较复杂,需要分情况讨论:
  % 1. 学术型硕士
  %    edegree:必须为Master of Arts或Master of Science(注意大小写)
  %             “哲学、文学、历史学、法学、教育学、艺术学门类,公共管理学科
  %              填写Master of Arts,其它填写Master of Science”
  %    emajor:“获得一级学科授权的学科填写一级学科名称,其它填写二级学科名称”
  % 2. 专业型硕士
  %    edegree:“填写专业学位英文名称全称”
  %    emajor:“工程硕士填写工程领域,其它专业学位不填写此项”
  % 3. 学术型博士
  %    edegree:Doctor of Philosophy(注意大小写)
  %    emajor:“获得一级学科授权的学科填写一级学科名称,其它填写二级学科名称”
  % 4. 专业型博士
  %    edegree:“填写专业学位英文名称全称”
  %    emajor:不填写此项
  edegree={Bachelor of Engineering Physics},
  emajor={Engineering Physics},
  eauthor={Wei, Wenyin},
  esupervisor={Professor Liang, Yunfeng},
  % eassosupervisor={Chen Wenguang},
  % 日期自动生成,若需指定按如下方式修改:
  % edate={December, 2005},
  %
  % 关键词用“英文逗号”分割
  ckeywords={扰动场, 边界局域模, 共振磁扰动, 高 m 线圈, 螺旋电流丝},
  ekeywords={magnetic perturbation field, edge localized mode (ELM), resonant magnetic perturbation (RMP),high m coil, helical current filament (HCL)}
}

% 定义中英文摘要和关键字
\begin{cabstract}
  本课题来自目前的先进托卡马克位型所面临的现实问题,尽管参数优良的 \Hmode 等离子体使得聚变达到所需参数目标有了更大的可能性,但同时也带来了新的问题。\Hmode 下等离子体边界高压力梯度和强电流密度蕴含的自由能,引起了边界局域模不稳定性。边界局域模会引起热负荷和粒子流强出现近似周期性的脉冲峰值,而这在 DEMO 堆中是不被允许的。

  为了抑制边界局域模, EAST 上先后测试了共振磁扰动线圈 RMP、高 m 线圈和低杂波驱动的螺旋电流丝,这三种扰动场产生机制有所差异,产生的效果也不尽相同。为了使扰动场相互配合达到最优的弱化乃至抑制边界局域模的效果,对它们在等离子体边界造成的扰动场协同作用的研究是很有必要的。 (1) 低 n 线圈,过去一般称为 RMP 线圈,布置在装置真空室内,它产生的低环向模数(1-4)的扰动场在 \east, \ddd 等托卡马克装置上实验验证了抑制边界局域模的效应。 (2) 高 m 线圈,是 EAST 团队近两年新设计的线圈,在等离子体环外加上一组四个的线圈,它的特征是扰动场环向模数 $n$ 分布较宽,极向模数 $m$ 较高,由于一组高 $m$ 线圈只分布在一个极向截面处,扰动场的局域性很强。(3) 低杂波驱动的螺旋电流丝,电流丝的具体物理机制还不甚明晰,但其亦能调节边界磁拓扑。由于低杂波天线不像共振磁扰动线圈在腔内易受到损坏且激发出的螺旋电流丝紧靠边界,它有望在 DEMO 堆及日后商业堆中灵活地调节磁拓扑结构。
  
  本文讨论了线圈之间如何配合以能够对等离子体施加合适的磁扰动场,以抑制 ELM 及不影响芯部等离子体作为主要判据,尝试建立了对磁扰动场的评估标准。进一步在第三章中通过磁力线追踪与扩散的技术给出了扰动场第一壁材料上的热负荷分布。我们主要依赖于对 ELM 的抑制效果来选择磁扰动场,而基于扰动场的热负荷调节也是扰动场的考量因素之一。%对下一代托卡马克而言,%\Hmode 等离子体会造成难以承受的热流和粒子流,
  % 扰动场可以此提供一种调节手段,避免脉冲式的 ELM 破裂造成的材料损害。

\end{cabstract}

% 如果习惯关键字跟在摘要文字后面,可以用直接命令来设置,如下:
% \ckeywords{\TeX, \LaTeX, CJK, 模板, 论文}

\begin{eabstract}
  The thesis discusses the realistic problem confronted by advanced tokamaks research. Though better confinement is obtained with \Hmode plasma than \Lmode, which makes it possible to achieve the threshold acquired by the fusion energy, new problems are also coming. The free energy stored in the high pressure gradient and strong current density in the edge of confined plasma induces edge localized mode (ELM) instability. ELM may cause too intense transient pulses of heat load and particle flux to sustain, which is not allowable in future tokamaks.

  In order to realize ELM suppression, multiple varieties of perturbation fields have been tested in EAST, \textit{i.e.} resonant magnetic perturbation coils, high $m$ coils and helical current filaments induced by lower hybrid wave. Each of these approaches has advantages and disadvantages. It is necessary to research on the collaborative effect if an optimal ELM suppression effect is anticipated. (1) Low n coils,also known as RMP coils, are distributed inside the vacuum vessel to induce the perturbant field with dominant low toroidal mode number $n=1-4$. Its effect to mitigate or suppress ELM is verified in \east, \ddd tokamaks \textit{etc.}. (2) High m coils are under design by \east team in these two years. Normally four coils are imposed in one poloidal cut $\phi=\text{const}$ with various theta. High $m$ coils have the characteristics that perturbant spectrum distribute widely in toroidal mode number $n$ while relatively high in poloidal mode number $m$. (3) Helical current filanments(HCFs) induced by lower hybrid waves(LHW). LHW is originally designed to drive core plasma current by Landau damping, but experimental evidence shows that there exists helical current filaments in the scrape-off layer (SOF) while LHW system switches on. Because of the fact that LHW antennas are shielded by limiters, therefore not easy to be damaged, and HCFs are close to the plasma edge, it has the potential to modify the magnetic topology near the edge of plasma flexibly in DEMO and next-generation tokamaks.

  The induced perturbant fields by above sources are discussed in chapter two to analyze their spectrum features, in which the Fourier spectrum $\tilde{b}^1_{mn}$ of the radial component of perturbation field near the edge of plasma and \Poincare plots are necessary to analyze the topology. How to acquire a satisfactory perturbant result by appropriate collaborative coils setup, \textit{i.e.} suppress ELM and sustain well confinement of plasma, constitutes the main content of chapter three. Furthermore, the heat load distribution patterns of various perturbant field combinations are analyzed in chapter four. Though we mainly rely on the ELM suppression effect to alter perturbant recipes, the possibility of adjustment of heat load distribution is considered to provide another perspective on the perturbant field. For next generation tokamaks, \Hmode plasma causes unafforable heat flux and particle flux pulses to the plasms-facing components, for which schemes to adjust the heat pattern are required.
\end{eabstract}

% \ekeywords{\TeX, \LaTeX, CJK, template, thesis}
% The thesis discusses the realistic problem confronted by advanced tokamaks research. Though better confinement is obtained with H-mode plasma than L-mode, which makes it possible to achieve the threshold acquired by the fusion energy, new problems are also coming. The free energy stored in the high pressure gradient and strong current density in the edge of confined plasma induces edge localized mode (ELM) instability. ELM may cause too intense transient pulses of heat load and particle flux to sustain, which is not allowable in future tokamaks.

% In order to realize ELM suppression, multiple varieties of perturbation fields have been tested in EAST, i.e. resonant magnetic perturbation coils, high $m$ coils and helical current filaments induced by lower hybrid wave. Each of these approaches has advantages and disadvantages. It is necessary to research on the collaborative effect if an optimal ELM suppression effect is anticipated. The coils involved in the research are (1) Low n coils,also known as RMP coils, (2) High m coils, (3) Helical current filanments(HCFs) induced by lower hybrid waves(LHW). 

% The induced perturbant fields by above sources are discussed in chapter two to analyze their spectrum features, in which the Fourier spectrum $\tilde{b}^1_{mn}$ of the radial component of perturbation field near the edge of plasma and Poincare plots are necessary to analyze the topology. How to acquire a satisfactory perturbant result by appropriate collaborative coils setup, i.e. suppress ELM and sustain well confinement of plasma, constitutes the main content of chapter three. Furthermore, the heat load distribution patterns of various perturbant field combinations are analyzed in chapter four. Though we mainly rely on the ELM suppression effect to alter perturbant recipes, the possibility of adjustment of heat load distribution is considered to provide another perspective on the perturbant field. 
