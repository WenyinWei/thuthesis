\thusetup{
  %******************************
  % 注意:
  %   1. 配置里面不要出现空行
  %   2. 不需要的配置信息可以删除
  %******************************
  %
  %=====
  % 秘级
  %=====
  secretlevel={秘密},
  secretyear={10},
  %
  %=========
  % 中文信息
  %=========
  ctitle={Vlasov-Poisson 系统的适定性},
  cdegree={},
  cdepartment={工程物理系},
  cmajor={数学与应用数学(第二学位)},
  cauthor={魏文崟},
  csupervisor={王学成 教授},
  % cassosupervisor={}, % 副指导老师
  ccosupervisor={于品 教授}, % 联合指导老师
  % 日期自动使用当前时间,若需指定按如下方式修改:
  % cdate={超新星纪元},
  %
  % 博士后专有部分
  catalognumber     = {分类号},  % 可以留空
  udc               = {UDC},  % 可以留空
  id                = {编号},  % 可以留空: id={},
  cfirstdiscipline  = {计算机科学与技术},  % 流动站(一级学科)名称
  cseconddiscipline = {系统结构},  % 专 业(二级学科)名称
  postdoctordate    = {2009 年 7 月——2011 年 7 月},  % 工作完成日期
  postdocstartdate  = {2009 年 7 月 1 日},  % 研究工作起始时间
  postdocenddate    = {2011 年 7 月 1 日},  % 研究工作期满时间
  %
  %=========
  % 英文信息
  %=========
  etitle={On the well-posedness of the Vlasov-Poisson system},
  % 这块比较复杂,需要分情况讨论:
  % 1. 学术型硕士
  %    edegree:必须为Master of Arts或Master of Science(注意大小写)
  %             “哲学、文学、历史学、法学、教育学、艺术学门类,公共管理学科
  %              填写Master of Arts,其它填写Master of Science”
  %    emajor:“获得一级学科授权的学科填写一级学科名称,其它填写二级学科名称”
  % 2. 专业型硕士
  %    edegree:“填写专业学位英文名称全称”
  %    emajor:“工程硕士填写工程领域,其它专业学位不填写此项”
  % 3. 学术型博士
  %    edegree:Doctor of Philosophy(注意大小写)
  %    emajor:“获得一级学科授权的学科填写一级学科名称,其它填写二级学科名称”
  % 4. 专业型博士
  %    edegree:“填写专业学位英文名称全称”
  %    emajor:不填写此项
  edegree={Doctor of Engineering},
  emajor={Computer Science and Technology},
  eauthor={Xue Ruini},
  esupervisor={Professor Zheng Weimin},
  eassosupervisor={Chen Wenguang},
  % 日期自动生成,若需指定按如下方式修改:
  % edate={December, 2005},
  %
  % 关键词用“英文逗号”分割
  ckeywords={Vlasov-Poisson, 全局存在性, 唯一性, 适定性},
  ekeywords={Vlasov-Poisson, global existence, uniqueness, well-posedness}
}

% 定义中英文摘要和关键字
\begin{cabstract}
  本文作为文献综述,总结了对 Vlasov-Poisson 系统所做的相关研究。Vlasov-Poisson 问题的适定性问题,即其解存在性与唯一性的证明,及解在时间上至多局部存在还是可以全局存在的问题。该综述的主体部分讲述了局部解的适定性问题,即通过光滑化原 Vlasov-Poisson 问题场函数的奇异性,得到逼近解在该修正逐渐弱化的极限下是一致收敛的结果,从而证明原 Vlasov-Poisson 问题的局部的适定性问题,从而证明局部解的存在性和唯一性。而等离子体物理领域著名的朗道阻尼现象,在本文介绍中也略有提及其相关的数学工作成果。

  本文主要围绕着 Vlasov-Poisson 系统的适定性问题,讨论了系统的存在性和唯一性问题。在第二章中我们采用了局部良态问题的求解方法之后,第三章和第四章分别给出了非相对论情形和相对论情形的全局解的存在性问题相关结果的整理。对于相对论的 Vlasov-Poisson 问题,在紧凑的veclocity支持下,证明了 $\mu=1$ 的情况具有全局存在性,而 $\mu=-1$  足够小的初始数据的情况具有全局存在性,而在较大的情况下则不然,解不能全局延拓。
  
  
\end{cabstract}

% 如果习惯关键字跟在摘要文字后面,可以用直接命令来设置,如下:
% \ckeywords{\TeX, \LaTeX, CJK, 模板, 论文}

\begin{eabstract}
Recent researches on the Vlasov-Poisson system have been concluded in this literature review. The well-posedness problem, when the Vlasov-Poisson system has a global and unique solution has been studied for a long time. The main part of the literature review contains the approximation method used to converge to a local-in-time solution, proving the local well-posedness.  The well-known Landau damping, a varitey of long-term time asymptotic behaviour of Vlasov-Poisson system, is introduced concisely before the main body.

The literature mainly centres on the topic of well-posedness in the Vlasov-Poisson system, discussing about the existence and uniqueness problem. After the method we used in Chap 2 to solve the local well-posedness problem, global solutions existence problem are presented in Chap 3 and Chap 4 respectively for non-relativistic and relativistic cases. For the relativistic Vlasov-Poisson problem, it is shown with compact veclocity support, that $\mu=1$ situations have global existence, while for $\mu=-1$ "small" enough cases are known to have global existence and a case of blow up with "large" enough spherically symmetric initial data. 

% Non-linear stability proof for the stable solutions in Vlasov-Poisson system are introduced in Sect. \ref{cha:stability}, mainly based on the energy-Casimir methoud which is an elegant method to settle stability issue in conservative systems. As the byproducts in the above proof, regularites about some concrete quantities, \textit{e.g.} $\vE$ and $\rho$, would be exhibited in Sect. \ref{cha:regularities}.

\end{eabstract}

% \ekeywords{\TeX, \LaTeX, CJK, template, thesis}
  