\chapter{课题介绍}

\section{EAST}
EAST (\textbf{E}xperimental \textbf{A}dvanced \textbf{S}uperconducting \textbf{T}okamak) 位于合肥等离子体物理研究所。

\section{边界局域模 Edge Localized Mode}
磁重联和它导致的磁场拓扑结构变化在聚变等离子体动力学中有着不可忽视的作用。由共振扰动场线圈引起的边界随机场被认为可以抑制在等离子体边界周期性或近似周期性的破裂 (crash)。边界局域模的这种破裂会导致和等离子体直接接触的材料受到极高的热负荷,并且在下一代的聚变设施中这样的热通量几乎是材料无法承受的。由 RMP 引发的随机场能够减少在等离子体边界由于压力梯度和电流而积蓄的自由能,从而避免边界局域模这种不稳定性的发生。

然后等离子体对扰动的响应往往会屏蔽掉 RMP 线圈施加的影响并且大大地降低磁场的随机程度,这使得通过 RMP 线圈能否有效可靠地抑制边界局域模 ELM 打上了问号。

边界局域模的削弱 mitigation,

边界局域模的抑制 suppression

目前学界对 ELM 的削弱和抑制之间的关键区别还不明晰,同时等离子体对 ELM 抑制的线性和非线性响应也有待探索。

\section{多种三维扰动场}
\subsection{共振磁扰动线圈 RMP}

一套 RMP 线圈系统于 2014 年安装在 EAST 的低场侧,它包含有两组线圈,一组八个。EAST 团队通过扰动场环向模数为 $n=1, 2$ 的 RMPs 实现了 \Rmnum{1} 型边界局域模的削弱和完全的抑制。
\subsection{高 m 线圈}

\subsection{由低杂波引发的螺旋电流丝 HCF}
RMP 有它致命的弱点,RMP 线圈置于腔内,这在 DEMO 堆的设计中是不被允许的。研究人员只能通过其他手段来改变边界磁拓扑。 
\subsubsection{低杂波加热手段简述}
低杂波加热原本用于芯部等离子体电流驱动,它通过朗道阻尼将动量传给等离子体,可以实现不依赖于离子回旋共振加热 (ICRH) 的长脉冲的 H 模运行,



\section{数值计算工具}
\subsection{偏微分方程求解}
偏微分方程的求解问题构成了现代工程领域许多重要的设计工作,不同计算框架和数值理论在各种高性能计算处理器的基础上组成了现代工业设计的重要工作——

\begin{itemize}
    \item \textit{SU2}, SU2 工具箱是 C++ 基础的软件工具开源集合,它的作用是偏微分方程的分析工作并求解在给定条件及偏微分方程的基础上进行设计优化。这套工具是用来为计算流体力学 Computational Fluid Dynamics (CFD) 和空气动力学形状优化而设计的,但它也能够进行扩展来处理任意几何的控制方程,例如位势流,弹性问题,电流力学问题,化学反应流以及其他问题。
    \item \textit{FEniCS}, FEniCS 是https://fenicsproject.org/ 
    The FEniCS computing platform
    FEniCS is a popular open-source (LGPLv3) computing platform for solving partial differential equations (PDEs). FEniCS enables users to quickly translate scientific models into efficient finite element code. With the high-level Python and C++ interfaces to FEniCS, it is easy to get started, but FEniCS offers also powerful capabilities for more experienced programmers. FEniCS runs on a multitude of platforms ranging from laptops to high-performance clusters.
    \item 等所采用的电磁场计算软件
\end{itemize}


\subsection{Particel-in-cell (PIC)}
line-tracing-flow code
在以上的偏微分方程求解时,物理问题允许将某一点(或一个邻域内)的物理量取其代表值来离散化,如有限体积法中取其网格内的平均值进行计算。然而,在并不一定完全服从高斯速度分布的等离子体物理研究中,这样的代表值很难抽取出来。类似的问题,在裂变堆中子物理计算中采用的是多群计算的方法;而在聚变等离子体物理问题中,不论是自然产生的等离子体还是人工产生的加速器,往往会出现相当奇异的各向异性的速度分布,Particle-in-cell, 即 PIC 粒子物理将电磁场进行常规的偏微分方程求解,另外,还令网格中分布着巨粒子。巨粒子对网格角点处的电磁场参数数值有所影响,同时巨粒子也会根据离散的电磁场计算其下一时间步长的速度和坐标,这样就一定程度上在完全的粒子模拟和有限网格计算方法之间达到所需要的性能、准确之间的平衡。

\section{Magnetohydrodynamics Instability (MHD instability)}
\subsection{Kink Mode}
\subsection{Balloning Mode}

\section{Edge Localize Mode (ELM)}
关于m边界局域模的理论,目前没有办法给我们定量的描述,关于到底能量和粒子损失速率有多快?于是和实验得到的输赢结果相比,就变成了不可能。然而我们可以比较的是实验观测到的时间尺度,例如,边界局域模的上升时间长度,持续长度以及在以及边界局域模重复频率的电话表现变化趋势,另外,边界局域模发声的镜像尺度上面向范围,仍然可以用,可以用,也可以用理论给出,而且可以和实验的发现进行一个对比。他有三类型的边界局域模就和正大周期性的边界局域模,嗯,强烈的不同,它有明显的,你学的性质在,太妃,类型的变局膜发生的时候,实际的输运,可以比在中间。



这两次跟着巨魔发生的之间的时间间隙。是由本日起,边界需要建立,压力梯度和电流密度切入的时间所决定的,直到这个稳定极限达到为止。





在这篇论文中做的对不同的边界局域模现象,用他们的物理本质进行了进行了描述。


值得注意的是。这样的物理描述不能作为一种对,边界局域模中蕴藏的复杂非线性物理,所做的绝对描述,然而,他和实验观测到的结果相匹配,并且是基于我们对磁流体稳定性分析的认识,基于我们目前的基本物理。

\begin{itemize}
    \item \textit{Dithering Cycles} 
    震荡周期型,在\Hmode 转换过程中,一个极限震荡周期,一种由于滞后的\Hmode ,功率阈值而产生的极限,震荡周期可能会发生。位于边界的压力和电流密度和他们在和等离子机处于l模状态时的姿势类似的,因此,震荡周期型不是一种典型的磁流体不稳定性,而是一种妖魔,lhl模转换系列
    \item \textit{Type III} 泰山类型的边界局域模,在l模转换之后,由于边界输运壁垒。边界的压力梯度和电流密度7度,变得更加强烈。他们是造成有主持流体不稳定性的,自由人的源头,这是在边界的电子温度不太高的情况下。这些不稳定性,是太白山类型的边界局域模,可能是一种复杂耦合的磁流体现象,高恩,阻尼不稳定性,如阻尼气球模和dn的kink膜,自由边界不稳定性造成的,他们可能迅速的加强书院,由于嘌呤核效应。在边界温度的增长。稳定性边界for the register to him,造成了边界局域模发生频率的下降,随着输入功率,对足够高的温度,阻尼效应不再扮演一个重要角色,而太姥山类型的边界局域模也被抑制掉了。
    \item \textit{Type I} 第3类型的第1类型的边界局域模,再高的边界温度的时候,第3类型的骗局魔已经被抑制住了,你李想的气球模限制了可以达到的边界,最大压力梯度,如果你想去求魔盒一种,dn的不稳定性耦合处了,很有可能又是一种,像cake不稳定是一样的不稳定性,由于高的边界电流密度,这样子,第一类型的边界局魔就会发展起来,如果通过等离子体形貌,等离子体的边界,可以达到第2种稳定的区间,气球模的区间,那么第1类型的边界局域模,就会被抑制住,而此时对边界压力梯度的限制,就将会有BN的磁流体不稳定性所决定。
\end{itemize}






边界局域模是一种在高\Hmode ,等离子体中的磁流体不稳定性,他是说,陡峭的边界温度7度以及密度梯度所造成的,他们将能量和粒子剧烈的从本义是边界中释放出去,扮演的重要角色是,提供一种稳定的\Hmode 饭店,但重复性的边界局域模发生的时候,并且可以帮助控制等离子体中的粒子存量。从还行,聚变等离子体中,边界局域现象,发生的情况来看,它们可以被清晰的划分为三种种类。

第1种。震荡周期型,由于分叉,由于\Hmode 的分差性质,在输入分界面的功率约等于。在震荡周期中发生的短暂的,l相位的女子体,他的馄饨的程度,并不显著的,比在低于阈值时的1妖魔高。

第3类型的边界局域模,边界这个边界局域模的重复频率,随着穿过分界面的人流。在超过阈值的时候,阈值功率的时候,也就是说重复频率对于输入功率的倒数为小于0,一个耦合的磁签前震荡他的环向魔术约等于5~10之间,极向魔术,在10~15之间被观测到了。在边界局域模发生的时候,有一个高水平的词圣诞,第3类型的边界局域模,在边界温度较高时高时是稳定的,这表明,他们和带阻尼的,磁流体不稳定性是有关的,第3类型,边界局域模的潜在物理机制候选,包括阻尼有主气球模,和一种全局不稳定相耦合,比如说自由边界模。

他唯一类型的边界局域模,这一种类型的边界局域模重复周期,随着穿过,分界面轮流的增强而增强,在目前的实践中,没有显著的耦合,此前震荡发生,被检测到,在这个边界局域模发生的过程中,有一种较高水平的,非耦合的磁阵站,没有吃的前兆可能是由于缺少检测,检测什么呢?检测磁流体魔术比较高的那些。嗯。。泰北一第一类型的边界局域模,可能是通过在等离子体边界到第二个稳定的气球区域,来稳定下来,然而理想的气球,摩的判据一般来说似乎似乎是一个必要条件,但不是一个充要条件,对于第1类型的边界局域模一个客,可能的候选人,对于第一类型的边界局域模来说事,于是是一种理想气球模和全局,李想,流体不稳定性之间的耦合,比如说你想听什么?边界局域模的描述仍然是定性的,显然,一个线性的稳定性分析,对于典型的电流和压力分布可以给出一个关于不稳定磁流体魔术的判据,然而理论的边界局域模研究仍然需要面对非线性的计算,他们才实际上解释了,在边界局势发生的时候受到增强的书院,总的来说这需要巨大的计算工作,但在未来应该是可及的,另外磁流体魔术的半径范围,定向范围使得边界局域模可以通过理论模型计算进行定量化并且和实验结果相比较在实验中一个更好的电信的关于稳定性边界的描述关于不同的边界局域模类型可能是是有可能的,这是由正在进行的努力关于诊断来决定温度分布,粒子密度分布电流密度有着高的时间和空间分辨率这些分布将会作为理论分析的输入。

​\section{主要目标}
本论文着重在通过模拟的手段对现有的多种三维磁场进行模拟仿真,他们的磁谱被设计用来削弱或者调节边界局域模的发生。但他们之间的同时作用可以如何达到对粒子束流和热流的调节作用。