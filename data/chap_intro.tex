\chapter{课题介绍}

\section{EAST}
EAST (\textbf{E}xperimental \textbf{A}dvanced \textbf{S}uperconducting \textbf{T}okamak) 位于合肥等离子体物理研究所。

\section{边界局域模 Edge Localized Mode}
磁重联和它导致的磁场拓扑结构变化在聚变等离子体动力学中有着不可忽视的作用。由共振扰动场线圈引起的边界随机场被认为可以抑制在等离子体边界周期性或近似周期性的破裂 (crash)。边界局域模的这种破裂会导致和等离子体直接接触的材料受到极高的热负荷,并且在下一代的聚变设施中这样的热通量几乎是材料无法承受的。由 RMP 引发的随机场能够减少在等离子体边界由于压力梯度和电流而积蓄的自由能,从而避免边界局域模这种不稳定性的发生。

然后等离子体对扰动的响应往往会屏蔽掉 RMP 线圈施加的影响并且大大地降低磁场的随机程度,这使得通过 RMP 线圈能否有效可靠地抑制边界局域模 ELM 打上了问号。

边界局域模的削弱 mitigation,

边界局域模的抑制 suppression

目前学界对 ELM 的削弱和抑制之间的关键区别还不明晰,同时等离子体对 ELM 抑制的线性和非线性响应也有待探索。

\section{多种三维扰动场}
\subsection{共振扰动场线圈 RMP}

一套 RMP 线圈系统于 2014 年安装在 EAST 的低场侧,它包含有两组线圈,一组八个。EAST 团队通过扰动场环向模数为 $n=1, 2$ 的 RMPs 实现了 \Rmnum{1} 型边界局域模的削弱和完全的抑制。
\subsection{高 m 线圈}

\subsection{由低杂波引发的螺旋电流丝 HCF}
RMP 有它致命的弱点,RMP 线圈置于腔内,这在 DEMO 堆的设计中是不被允许的。研究人员只能通过其他手段来改变边界磁拓扑。 
\subsubsection{低杂波加热手段简述}
低杂波加热原本用于芯部等离子体电流驱动,它通过朗道阻尼将动量传给等离子体,可以实现不依赖于离子回旋共振加热 (ICRH) 的长脉冲的 H 模运行,