% 如果使用声明扫描页,将可选参数指定为扫描后的 PDF 文件名,例如:
% \begin{acknowledgement}[scan-statement.pdf]
\begin{acknowledgement}
  衷心感谢导师梁云峰教授给我的毕设选题,它在原理上不是特别困难,是让我熟悉磁拓扑结构很好的练手。其基础的 ERGOS 程序的架构问题确实已经造成了相关研究者很长时间的困扰。从我收到的师兄的反馈而言,重构 ERGOS 的愿望确实较为强烈,于是我的毕设过程中便将其重构了一遍以便于并行化。也许对于单个线圈的分析工作还可以忍受原本的程序架构,但一旦涉及到优化问题中的十余个线圈的协同模拟则会让人感到头疼。梁老师常常抽时间和我一聊就是两三个小时,解答我的疑惑并跟进课题进展。此次所幸题目主要是模拟工作,疫情造成的影响主要在生活琐事上的,与毕设实验的同学相比起来算是幸运了不少。
  
  感谢本科期间高喆教授给我的诸多帮助,不论是本科的学习生活还是研修经历,高喆教授都帮了我许多。我已经占了高老师办公室将近一个年头的工位了,虽然有些不舍得但今年就要搬走了。感谢高喆老师在本科期间对我各种冒失的宽容和支持。谭熠老师在每次组会上也常常和我交流,感谢两位老师在组会上给出的意见。
  
  感谢各位等离子体所认识的师兄师姐,大家在我问问题的时候几乎知无不言,很快就能给我反馈。特别要感谢贾曼妮师姐和张华祥师兄,两位都曾在我的毕设相关方面做过研究,给了我一定程度的帮助。同时还要感谢廖亮师兄和刘少承老师,疫情期间他们常常与我沟通等离子体所的具体情况。

  感谢在全国高性能计算大赛中认识的比赛组织者,中国信通院的郑立同学。文章中的计算任务大部分都是在他提供的服务器上进行运算的,希望我们以后还能长久合作。

  感谢我的家人,如我的二伯母、母亲等亲人,在疫情期间给了我尽可能的支持,起先于湖北家乡过年的日子里确实是非常困难,但所幸终是过去了,那段时间基本都是靠乡里亲人接济着。后来疫情缓解后随父母回到广东,那时国内境况便好了很多。家人的支持使得我能够持之以恒地完成论文,如果没有他们的支持的话这篇论文着实难以为继。


  % 对磁拓扑结构的研究是个可深可浅的课题,这篇文章受限于时间原因主要讨论的是真空场的情况,期待结题后的暑期能够结合磁流体引擎对等离子体反馈做出更进一步的研究。
\end{acknowledgement}
