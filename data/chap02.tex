\chapter{Stability}
\label{cha:stability}
Large amount of attention has been put on the problem of stability of stationary solutions of the Vlasov-Poisson system, both in the stellar dynamics and the plasma physics cases. The energy-Casimir 
method has been used to prove non-linear stability for various conservative systems, and \cite{rein_non-linear_1994} employed the method to prove non-linear stability of the Vlasov-Poisson system in three grometrically different setting. The three settings are the situations where the ion density is replaced by a given fixed ion background, the plasma can be spatially periodic, or can be restricted to a bounded domain. 


With the exception of the first case, stationary solutions exist in these settings and also in the stellar dynamic case. 
In the physics literature there are numerous investigations of the problem of stability of these stationary solutions, both linear and non-linear, and we refer to [1,2]\footnote{Not yet found} and the monographs [4,5]\footnote{Not yet found} for references. In contrast, very little rigorous mathematics exists on this problem. In [2,9]\footnote{Not yet found} non-linear stability of stationary solutions in a spatially periodic, plasma physics setting is established for the Vlasov-Poisson system and the 
relativistic Vlasov-Maxwell system, respectively, see also [lo, 111\footnote{Not yet found}. The problem of linear stability is investigated in [l]\footnote{Not yet found}, both for the plasma physics and the stellar dynamics cases. The phenomenon of Landau damping is established mathematically 
in [ 6 ]\footnote{Not yet found} for the one-dimensional, linearized Vlasov-Poisson system. 
The starting point of the present investigation is a general method to assert 
non-linear stability of stationary solutions for (infinite-dimensional, degenerate) 
Hamiltonian systems, which is presented in [S]\footnote{Not yet found}. We briefly review this method. Let the 
system under consideration be described by the equation of motion 

$$\dot{u} = A(u)$$

on some state space $X , A : D(A) \rightarrow X $ a (non-linear) operator, and let $u_0$ be the 
stationary solution whose stability we want to investigate. The following steps lead to 
a stability result for $u_0$:

\begin{enumerate}
  \item Find the energy (Hamiltonian) $H :X + \bbR$ of the system; $d H ( u ( t ) )/dt = 0$ along 
  solutions. 
  \item Relate $u_0$ to another conserved quantity Casimir functional $C : X + \bbR$ such that $u_0$ is a critical point of $H_C := H + C$, i.e. $DH_C(u_0) = 0$. 
  \item Show that the quadratic part in the expansion of $H_c$ at $u_0$ 
  $$H_C(u) = H_C(u_00) + DH_C(u_0)(u - u_0) + D^2H_C(u_0)(u - u_0, u - u_0) + ... $$
  is positive definite, more precisely, find a norm $|| \cdot ||_a $ on X such that 
  $$H_C(u) - H_C(u_0) - DH_C(u_0)(U - u_0) \geq  C||u-u_0||^2_a \in X,$$ 
  for some $C > 0$. 
  \item Find a norm $||\cdot ||_b$ on $X$ with respect to which $H_c$ is continuous at $u_0$. 
\end{enumerate}


If Steps (1)-(3) can be carried through, then for any solution 

and with Step (4) we conclude that for any $\varepsilon > 0$ there exists $\delta > 0$ such that $||u(0)-u_0||_b < \delta$ implies  $$\left\|u(t)-u_{0}\right\|^{2}_a \leqslant \frac{1}{C}\left|H_{C}(u(0))-H_{C}\left(u_{0}\right)\right| ,$$ \textit{i.e}. $u_0$ is non-linearly stable. 

Though the energy-Casimir method is elegant and appealing, in [S] 
it is pointed out that the appearance of large velocities could cause the method to run into trouble.
