\chapter{带 English 的标题}
\label{cha:intro}

\section{定理环境}
\label{sec:theorem}

给大家演示一下各种和证明有关的环境:

\begin{assumption}
待月西厢下,迎风户半开;隔墙花影动,疑是玉人来。
\begin{align}
  \label{eq:eqnxmp}
  c & = a^2 - b^2 \\
    & = (a+b)(a-b)
\end{align}
\end{assumption}

千辛万苦,历尽艰难,得有今日。然相从数千里,未曾哀戚。今将渡江,方图百年欢笑,如
何反起悲伤?(引自《杜十娘怒沉百宝箱》)

\begin{definition}
子曰:「道千乘之国,敬事而信,节用而爱人,使民以时。」
\end{definition}


\begin{proposition}
 曾子曰:「吾日三省吾身 —— 为人谋而不忠乎?与朋友交而不信乎?传不习乎?」
\end{proposition}


\begin{remark}
天不言自高,水不言自流。
\begin{gather*}
\begin{split}
\varphi(x,z)
&=z-\gamma_{10}x-\gamma_{mn}x^mz^n\\
&=z-Mr^{-1}x-Mr^{-(m+n)}x^mz^n
\end{split}\\[6pt]
\begin{align} \zeta^0&=(\xi^0)^2,\\
\zeta^1 &=\xi^0\xi^1,\\
\zeta^2 &=(\xi^1)^2,
\end{align}
\end{gather*}
\end{remark}


\begin{axiom}
两点间直线段距离最短。
\begin{align}
x&\equiv y+1\pmod{m^2}\\
x&\equiv y+1\mod{m^2}\\
x&\equiv y+1\pod{m^2}
\end{align}
\end{axiom}

《彖曰》:大哉乾元,万物资始,乃统天。云行雨施,品物流形。大明始终,六位时成,时
乘六龙以御天。乾道变化,各正性命,保合大和,乃利贞。首出庶物,万国咸宁。

《象曰》:天行健,君子以自强不息。潜龙勿用,阳在下也。见龙再田,德施普也。终日乾
乾,反复道也。或跃在渊,进无咎也。飞龙在天,大人造也。亢龙有悔,盈不可久也。用九,
天德不可为首也。   

\newcommand\dif{\mathop{}\!\mathrm{d}}
\begin{lemma}
《猫和老鼠》是我最爱看的动画片。
\begin{multline*}%\tag*{[a]} % 这个不出现在索引中
\int_a^b\biggl\{\int_a^b[f(x)^2g(y)^2+f(y)^2g(x)^2]
 -2f(x)g(x)f(y)g(y)\dif x\biggr\}\dif y \\
 =\int_a^b\biggl\{g(y)^2\int_a^bf^2+f(y)^2
  \int_a^b g^2-2f(y)g(y)\int_a^b fg\biggr\}\dif y
\end{multline*}
\end{lemma}

行行重行行,与君生别离。相去万余里,各在天一涯。道路阻且长,会面安可知。胡马依北
风,越鸟巢南枝。相去日已远,衣带日已缓。浮云蔽白日,游子不顾返。思君令人老,岁月
忽已晚。  弃捐勿复道,努力加餐饭。

\begin{theorem}\label{the:theorem1}
犯我强汉者,虽远必诛\hfill —— 陈汤(汉)
\end{theorem}
\begin{subequations}
\begin{align}
y & = 1 \\
y & = 0
\end{align}
\end{subequations}
道可道,非常道。名可名,非常名。无名天地之始;有名万物之母。故常无,欲以观其妙;
常有,欲以观其徼。此两者,同出而异名,同谓之玄。玄之又玄,众妙之门。上善若水。水
善利万物而不争,处众人之所恶,故几于道。曲则全,枉则直,洼则盈,敝则新,少则多,
多则惑。人法地,地法天,天法道,道法自然。知人者智,自知者明。胜人者有力,自胜
者强。知足者富。强行者有志。不失其所者久。死而不亡者寿。

\begin{proof}
燕赵古称多感慨悲歌之士。董生举进士,连不得志于有司,怀抱利器,郁郁适兹土,吾
知其必有合也。董生勉乎哉?

夫以子之不遇时,苟慕义强仁者,皆爱惜焉,矧燕、赵之士出乎其性者哉!然吾尝闻
风俗与化移易,吾恶知其今不异于古所云邪?聊以吾子之行卜之也。董生勉乎哉?

吾因子有所感矣。为我吊望诸君之墓,而观于其市,复有昔时屠狗者乎?为我谢
曰:“明天子在上,可以出而仕矣!” \hfill —— 韩愈《送董邵南序》
\end{proof}

\begin{corollary}
  四川话配音的《猫和老鼠》是世界上最好看最好听最有趣的动画片。
\begin{alignat}{3}
V_i & =v_i - q_i v_j, & \qquad X_i & = x_i - q_i x_j,
 & \qquad U_i & = u_i,
 \qquad \text{for $i\ne j$;}\label{eq:B}\\
V_j & = v_j, & \qquad X_j & = x_j,
  & \qquad U_j & u_j + \sum_{i\ne j} q_i u_i.
\end{alignat}
\end{corollary}

迢迢牵牛星,皎皎河汉女。
纤纤擢素手,札札弄机杼。
终日不成章,泣涕零如雨。
河汉清且浅,相去复几许。
盈盈一水间,脉脉不得语。

\begin{example}
  大家来看这个例子。
\begin{equation}
  \label{ktc}
  \begin{cases}
    \nabla f(\bm{x}^*) - \sum_{j=1}^p \lambda_j \nabla g_j(\bm{x}^*)
      = 0 \\[0.3cm]
    \lambda_j g_j(\bm{x}^*) = 0, \quad j = 1, 2, \dots, p \\[0.2cm]
    \lambda_j \ge 0, \quad j = 1, 2, \dots, p.
  \end{cases}
\end{equation}
\end{example}

\begin{exercise}
  请列出 Andrew S. Tanenbaum 和 W. Richard Stevens 的所有著作。
\end{exercise}

\begin{conjecture} \textit{Poincare Conjecture} If in a closed three-dimensional
  space, any closed curves can shrink to a point continuously, this space can be
  deformed to a sphere.
\end{conjecture}

\begin{problem}
 回答还是不回答,是个问题。
\end{problem}

如何引用定理~\ref{the:theorem1} 呢?加上 \cs{label} 使用 \cs{ref} 即可。妾发
初覆额,折花门前剧。郎骑竹马来,绕床弄青梅。同居长干里,两小无嫌猜。 十四为君妇,
羞颜未尝开。低头向暗壁,千唤不一回。十五始展眉,愿同尘与灰。常存抱柱信,岂上望夫
台。 十六君远行,瞿塘滟滪堆。五月不可触,猿声天上哀。门前迟行迹,一一生绿苔。苔深
不能扫,落叶秋风早。八月蝴蝶来,双飞西园草。感此伤妾心,坐愁红颜老。

\section{参考文献}
\label{sec:bib}
当然参考文献可以直接写 \cs{bibitem},虽然费点功夫,但是好控制,各种格式可以自己随意改
写。

本模板推荐使用 BIB\TeX,分别提供数字引用(\texttt{thuthesis-numeric.bst})和作
者年份引用(\texttt{thuthesis-author-year.bst})样式,基本符合学校的参考文献格式
(如专利等引用未加详细测试)。看看这个例子,关于书的~\cite{tex, companion,
  ColdSources},还有这些~\cite{Krasnogor2004e, clzs, zjsw},关于杂志
的~\cite{ELIDRISSI94, MELLINGER96, SHELL02},硕士论文~\cite{zhubajie,
  metamori2004},博士论文~\cite{shaheshang, FistSystem01},标准文
件~\cite{IEEE-1363},会议论文~\cite{DPMG,kocher99},技术报告~\cite{NPB2},电子文
献~\cite{chuban2001,oclc2000}。
若使用著者-出版年制,中文参考文献~\cite{cnarticle}应增加
\texttt{key=\{pinyin\}} 字段,以便正确进行排序~\cite{cnproceed}。
另外,如果对参考文献有不如意的地方,请手动修改 \texttt{bbl} 文件。

有时候不想要上标,那么可以这样~\inlinecite{shaheshang},这个非常重要。

有时候一些参考文献没有纸质出处,需要标注 URL。缺省情况下,URL 不会在连字符处断行,
这可能使得用连字符代替空格的网址分行很难看。如果需要,可以将模板类文件中
\begin{verbatim}
\RequirePackage{hyperref}
\end{verbatim}
一行改为:
\begin{verbatim}
\PassOptionsToPackage{hyphens}{url}
\RequirePackage{hyperref}
\end{verbatim}
使得连字符处可以断行。更多设置可以参考 \texttt{url} 宏包文档。

\section{公式}
\label{sec:equation}
\renewcommand\vec{\symbf}
\newcommand\mat{\symbf}
贝叶斯公式如式~(\ref{equ:chap1:bayes}),其中 $p(y|\vec{x})$ 为后验;
$p(\vec{x})$ 为先验;分母 $p(\vec{x})$ 为归一化因子。
\begin{equation}
\label{equ:chap1:bayes}
p(y|\vec{x}) = \frac{p(\vec{x},y)}{p(\vec{x})}=
\frac{p(\vec{x}|y)p(y)}{p(\vec{x})}
\end{equation}

论文里面公式越多,\TeX{} 就越 happy。再看一个 \pkg{amsmath} 的例子:
\newcommand{\envert}[1]{\left\lvert#1\right\rvert}
\begin{equation}\label{detK2}
\det\mat{K}(t=1,t_1,\dots,t_n)=\sum_{I\in\vec{n}}(-1)^{\envert{I}}
\prod_{i\in I}t_i\prod_{j\in I}(D_j+\lambda_jt_j)\det\vec{A}
^{(\lambda)}(\overline{I}|\overline{I})=0.
\end{equation}

前面定理示例部分列举了很多公式环境,可以说把常见的情况都覆盖了,大家在写公式的时
候一定要好好看 \pkg{amsmath} 的文档,并参考模板中的用法:
\begin{multline*}%\tag{[b]} % 这个出现在索引中的
\int_a^b\biggl\{\int_a^b[f(x)^2g(y)^2+f(y)^2g(x)^2]
 -2f(x)g(x)f(y)g(y)\dif x\biggr\}\dif y \\
 =\int_a^b\biggl\{g(y)^2\int_a^bf^2+f(y)^2
  \int_a^b g^2-2f(y)g(y)\int_a^b fg\biggr\}\dif y
\end{multline*}

其实还可以看看这个多级规划:
\begin{equation}
  \label{bilevel}
  \begin{cases}
    \max_{\bm{x}} F(\bm{x}, y_1^*, y_2^*, \dots, y_m^*) \\
      \text{subject to:} \\
      \qquad G(\bm{x}) \le 0 \\
      \qquad (y_1^*, y_2^*, \dots, y_m^*) \text{ solves problems }
        (i = 1, 2, \dots, m) \\
      \qquad
        \begin{cases}
          \max_{\bm{x}} f_i(\bm{x}, y_1, y_2, \dots, y_m) \\
          \text{subject to:} \\
          \qquad g_i(\bm{x}, y_1, y_2, \dots, y_m) \le 0.
        \end{cases}
  \end{cases}
\end{equation}
这些跟规划相关的公式都来自于刘宝碇老师《不确定规划》的课件。
