\chapter{外文资料原文}
\label{cha:engorg}

\title{The title of the English paper}

\textbf{Abstract:} As one of the most widely used techniques in operations
research, \emph{ mathematical programming} is defined as a means of maximizing a
quantity known as \emph{bjective function}, subject to a set of constraints
represented by equations and inequalities. Some known subtopics of mathematical
programming are linear programming, nonlinear programming, multiobjective
programming, goal programming, dynamic programming, and multilevel
programming$^{[1]}$.

It is impossible to cover in a single chapter every concept of mathematical
programming. This chapter introduces only the basic concepts and techniques of
mathematical programming such that readers gain an understanding of them
throughout the book$^{[2,3]}$.


\section{Single-Objective Programming}
The general form of single-objective programming (SOP) is written
as follows,
\begin{equation*} % 如果附录中的公式不想让它出现在公式索引中,那就请
                             % 用 equation*
\left\{\begin{array}{l}
\max \,\,f(x)\\[0.1 cm]
\mbox{subject to:} \\ [0.1 cm]
\qquad g_j(x)\le 0,\quad j=1,2,\cdots,p
\end{array}\right.
\end{equation*}
which maximizes a real-valued function $f$ of
$x=(x_1,x_2,\cdots,x_n)$ subject to a set of constraints.

\newcommand\Real{\mathbf{R}}
\newtheorem{mpdef}{Definition}[chapter]
\begin{mpdef}
In SOP, we call $x$ a decision vector, and
$x_1,x_2,\cdots,x_n$ decision variables. The function
$f$ is called the objective function. The set
\begin{equation*}
S=\left\{x\in\Real^n\bigm|g_j(x)\le 0,\,j=1,2,\cdots,p\right\}
\end{equation*}
is called the feasible set. An element $x$ in $S$ is called a
feasible solution.
\end{mpdef}

\newtheorem{mpdefop}[mpdef]{Definition}
\begin{mpdefop}
A feasible solution $x^*$ is called the optimal
solution of SOP if and only if
\begin{equation}
f(x^*)\ge f(x)
\end{equation}
for any feasible solution $x$.
\end{mpdefop}

One of the outstanding contributions to mathematical programming was known as
the Kuhn-Tucker conditions\ref{eq:ktc}. In order to introduce them, let us give
some definitions. An inequality constraint $g_j(x)\le 0$ is said to be active at
a point $x^*$ if $g_j(x^*)=0$. A point $x^*$ satisfying $g_j(x^*)\le 0$ is said
to be regular if the gradient vectors $\nabla g_j(x)$ of all active constraints
are linearly independent.

Let $x^*$ be a regular point of the constraints of SOP and assume that all the
functions $f(x)$ and $g_j(x),j=1,2,\cdots,p$ are differentiable. If $x^*$ is a
local optimal solution, then there exist Lagrange multipliers
$\lambda_j,j=1,2,\cdots,p$ such that the following Kuhn-Tucker conditions hold,
\begin{equation}
\label{eq:ktc}
\left\{\begin{array}{l}
    \nabla f(x^*)-\sum\limits_{j=1}^p\lambda_j\nabla g_j(x^*)=0\\[0.3cm]
    \lambda_jg_j(x^*)=0,\quad j=1,2,\cdots,p\\[0.2cm]
    \lambda_j\ge 0,\quad j=1,2,\cdots,p.
\end{array}\right.
\end{equation}
If all the functions $f(x)$ and $g_j(x),j=1,2,\cdots,p$ are convex and
differentiable, and the point $x^*$ satisfies the Kuhn-Tucker conditions
(\ref{eq:ktc}), then it has been proved that the point $x^*$ is a global optimal
solution of SOP.

\subsection{Linear Programming}
\label{sec:lp}

If the functions $f(x),g_j(x),j=1,2,\cdots,p$ are all linear, then SOP is called
a {\em linear programming}.

The feasible set of linear is always convex. A point $x$ is called an extreme
point of convex set $S$ if $x\in S$ and $x$ cannot be expressed as a convex
combination of two points in $S$. It has been shown that the optimal solution to
linear programming corresponds to an extreme point of its feasible set provided
that the feasible set $S$ is bounded. This fact is the basis of the {\em simplex
  algorithm} which was developed by Dantzig as a very efficient method for
solving linear programming.
\begin{table}[ht]
\centering
  \centering
  \caption*{Table~1\hskip1em This is an example for manually numbered table, which
    would not appear in the list of tables}
  \label{tab:badtabular2}
  \begin{tabular}[c]{|m{1.5cm}|c|c|c|c|c|c|}\hline
    \multicolumn{2}{|c|}{Network Topology} & \# of nodes &
    \multicolumn{3}{c|}{\# of clients} & Server \\\hline
    GT-ITM & Waxman Transit-Stub & 600 &
    \multirow{2}{2em}{2\%}&
    \multirow{2}{2em}{10\%}&
    \multirow{2}{2em}{50\%}&
    \multirow{2}{1.2in}{Max. Connectivity}\\\cline{1-3}
    \multicolumn{2}{|c|}{Inet-2.1} & 6000 & & & &\\\hline
    \multirow{2}{1.5cm}{Xue} & Rui  & Ni &\multicolumn{4}{c|}{\multirow{2}*{\thuthesis}}\\\cline{2-3}
    & \multicolumn{2}{c|}{ABCDEF} &\multicolumn{4}{c|}{} \\\hline
\end{tabular}
\end{table}

\subsection{Nonlinear Programming}

If at l EAST  one of the functions $f(x),g_j(x),j=1,2,\cdots,p$ is nonlinear, then
SOP is called a {\em nonlinear programming}.

A large number of classical optimization methods have been developed to treat
special-structural nonlinear programming based on the mathematical theory
concerned with analyzing the structure of problems.
\begin{figure}[h]
  \centering
  \includegraphics{thu-lib-logo.pdf}
  \caption*{Figure~1\quad This is an example for manually numbered figure,
    which would not appear in the list of figures}
  \label{tab:badfigure2}
\end{figure}

Now we consider a nonlinear programming which is confronted solely with
maximizing a real-valued function with domain $\Real^n$.  Whether derivatives are
available or not, the usual strategy is first to select a point in $\Real^n$ which
is thought to be the most likely place where the maximum exists. If there is no
information available on which to base such a selection, a point is chosen at
random. From this first point an attempt is made to construct a sequence of
points, each of which yields an improved objective function value over its
predecessor. The next point to be added to the sequence is chosen by analyzing
the behavior of the function at the previous points. This construction continues
until some termination criterion is met. Methods based upon this strategy are
called {\em ascent methods}, which can be classified as {\em direct methods},
{\em gradient methods}, and {\em Hessian methods} according to the information
about the behavior of objective function $f$. Direct methods require only that
the function can be evaluated at each point. Gradient methods require the
evaluation of first derivatives of $f$. Hessian methods require the evaluation
of second derivatives. In fact, there is no superior method for all
problems. The efficiency of a method is very much dependent upon the objective
function.

\subsection{Integer Programming}

{\em Integer programming} is a special mathematical programming in which all of
the variables are assumed to be only integer values. When there are not only
integer variables but also conventional continuous variables, we call it {\em
  mixed integer programming}. If all the variables are assumed either 0 or 1,
then the problem is termed a {\em zero-one programming}. Although integer
programming can be solved by an {\em exhaustive enumeration} theoretically, it
is impractical to solve realistically sized integer programming problems. The
most successful algorithm so far found to solve integer programming is called
the {\em branch-and-bound enumeration} developed by Balas (1965) and Dakin
(1965). The other technique to integer programming is the {\em cutting plane
  method} developed by Gomory (1959).

\hfill\textit{Uncertain Programming\/}\quad(\textsl{BaoDing Liu, 2006.2})

\section*{References}
\noindent{\itshape NOTE: These references are only for demonstration. They are
  not real citations in the original text.}

\begin{translationbib}
\item Donald E. Knuth. The \TeX book. Addison-Wesley, 1984. ISBN: 0-201-13448-9
\item Paul W. Abrahams, Karl Berry and Kathryn A. Hargreaves. \TeX\ for the
  Impatient. Addison-Wesley, 1990. ISBN: 0-201-51375-7
\item David Salomon. The advanced \TeX book.  New York : Springer, 1995. ISBN:0-387-94556-3
\end{translationbib}

\chapter{外文资料的调研阅读报告或书面翻译}

\title{英文资料的中文标题}

{\heiti 摘要:} 本章为外文资料翻译内容。如果有摘要可以直接写上来,这部分好像没有
明确的规定。

\section{单目标规划}
北冥有鱼,其名为鲲。鲲之大,不知其几千里也。化而为鸟,其名为鹏。鹏之背,不知其几
千里也。怒而飞,其翼若垂天之云。是鸟也,海运则将徙于南冥。南冥者,天池也。
\begin{equation}\tag*{(123)}
 p(y|\mathbf{x}) = \frac{p(\mathbf{x},y)}{p(\mathbf{x})}=
\frac{p(\mathbf{x}|y)p(y)}{p(\mathbf{x})}
\end{equation}

吾生也有涯,而知也无涯。以有涯随无涯,殆已!已而为知者,殆而已矣!为善无近名,为
恶无近刑,缘督以为经,可以保身,可以全生,可以养亲,可以尽年。

\subsection{线性规划}
庖丁为文惠君解牛,手之所触,肩之所倚,足之所履,膝之所倚,砉然响然,奏刀騞然,莫
不中音,合于桑林之舞,乃中经首之会。
\begin{table}[ht]
\centering
  \centering
  \caption*{表~1\hskip1em 这是手动编号但不出现在索引中的一个表格例子}
  \label{tab:badtabular3}
  \begin{tabular}[c]{|m{1.5cm}|c|c|c|c|c|c|}\hline
    \multicolumn{2}{|c|}{Network Topology} & \# of nodes &
    \multicolumn{3}{c|}{\# of clients} & Server \\\hline
    GT-ITM & Waxman Transit-Stub & 600 &
    \multirow{2}{2em}{2\%}&
    \multirow{2}{2em}{10\%}&
    \multirow{2}{2em}{50\%}&
    \multirow{2}{1.2in}{Max. Connectivity}\\\cline{1-3}
    \multicolumn{2}{|c|}{Inet-2.1} & 6000 & & & &\\\hline
    \multirow{2}{1.5cm}{Xue} & Rui  & Ni &\multicolumn{4}{c|}{\multirow{2}*{\thuthesis}}\\\cline{2-3}
    & \multicolumn{2}{c|}{ABCDEF} &\multicolumn{4}{c|}{} \\\hline
\end{tabular}
\end{table}

文惠君曰:“嘻,善哉!技盖至此乎?”庖丁释刀对曰:“臣之所好者道也,进乎技矣。始臣之
解牛之时,所见无非全牛者;三年之后,未尝见全牛也;方今之时,臣以神遇而不以目视,
官知止而神欲行。依乎天理,批大郤,导大窾,因其固然。技经肯綮之未尝,而况大坬乎!
良庖岁更刀,割也;族庖月更刀,折也;今臣之刀十九年矣,所解数千牛矣,而刀刃若新发
于硎。彼节者有间而刀刃者无厚,以无厚入有间,恢恢乎其于游刃必有余地矣。是以十九年
而刀刃若新发于硎。虽然,每至于族,吾见其难为,怵然为戒,视为止,行为迟,动刀甚微,
謋然已解,如土委地。提刀而立,为之而四顾,为之踌躇满志,善刀而藏之。”

文惠君曰:“善哉!吾闻庖丁之言,得养生焉。”


\subsection{非线性规划}
孔子与柳下季为友,柳下季之弟名曰盗跖。盗跖从卒九千人,横行天下,侵暴诸侯。穴室枢
户,驱人牛马,取人妇女。贪得忘亲,不顾父母兄弟,不祭先祖。所过之邑,大国守城,小
国入保,万民苦之。孔子谓柳下季曰:“夫为人父者,必能诏其子;为人兄者,必能教其弟。
若父不能诏其子,兄不能教其弟,则无贵父子兄弟之亲矣。今先生,世之才士也,弟为盗
跖,为天下害,而弗能教也,丘窃为先生羞之。丘请为先生往说之。”
\begin{figure}[h]
  \centering
  \includegraphics{thu-whole-logo.pdf}
  \caption*{图~1\hskip1em 这是手动编号但不出现索引中的图片的例子}
  \label{tab:badfigure3}
\end{figure}

柳下季曰:“先生言为人父者必能诏其子,为人兄者必能教其弟,若子不听父之诏,弟不受
兄之教,虽今先生之辩,将奈之何哉?且跖之为人也,心如涌泉,意如飘风,强足以距敌,
辩足以饰非。顺其心则喜,逆其心则怒,易辱人以言。先生必无往。”

孔子不听,颜回为驭,子贡为右,往见盗跖。


\title{ EAST  托卡马克上低杂波引起的磁拓扑变化,及其对边界局域模的显著影响}

{\heiti 摘要:} 当低杂波和离子回旋共振加热作用在 H-mode 的等离子体,在  EAST  上观测到了强烈的减弱边界局域模的作用。这种效果是由于低杂波引起的螺旋电流丝沿着磁力线在刮削层中不断流动的效果。和共轭磁扰动的效果类似,在低杂波运作期间也出现了束流在偏滤器上落点分裂的现象。通过在磁力线追踪程序中加入螺旋电流丝,本文也定性地模拟了其对磁拓扑结构的改变的作用。

% TODO Add figure & citation


对聚变能源研究及相关技术的巨大挑战来自于如何将炽热的等离子体约束住,使得接触等离子体的材料在运行期间维持一个可以接受(稳态和瞬态)的热负荷及粒子束流强度。
当托卡马克中的等离子体工作在高约束(\Hmode)状态的时候,等离子体能量约束时间显著增长。然而其后果则是等离子体边界上压强有着更大的梯度,连带着还有边界上增强了的电流密度,它可以超过阈值以驱动磁流体不稳定性,这被称为边界局域模。边界局域模会导致近似周期性的大量能量和粒子从本应受约束的区域损失,同时又会导致对接触的等离子体材料的严重损害,下一代的聚变设备,如 ITER 和 DEMO 装置,需要一种可靠的手段来控制或者抑制剧烈的边界区域模。

共轭磁扰动(RMP)改变了等离子体的磁拓扑结构,已经被用在 DIII-D 装置内完全抑制边界局域模,或者在实验中,抑制边界区域,这个的意思是增加边界局域模的频率,而减低每一次边界局域模发生的幅度,must和ORG装置上面得到实验。尽管这个物理机制还不是很清楚,。不同装置上得到的实验结果都表明是拓扑,有着一个很关键的作用,在整体约束中,以及边界磁流体稳定性,何等的相互作用,特别是对于,偏滤器


目前来说,在所有现存的以共轭磁扰动减弱或抑制边界局域模的实验中,磁扰动均由腔内或者腔外的线圈系统所激发。腔内的磁扰动线圈已在 ITER 设计上被考虑并做出了设计,用于抑制边界局域模的发生。但在未来的聚变反应堆中,(DEMO)腔内的磁扰动线圈可能不现实。于是通过其他机制改变磁拓扑来控制边界局域模,对于下一代的托克马克提供了一个有吸引力的解决方案。

最近 EAST 上面的研究结果表明,低杂波和共轭磁扰动的效果类似,通过改变磁拓扑,可以作为一种有效的减弱或者抑制边界局域模的手段。这篇快报阐述了低杂波对边界局域模表现的影响及偏滤器平板上的热负荷分布;同时记录了低杂波驱动下产生的,在刮削层中沿着磁力线流动的螺旋电流丝的实验结果(该螺旋电流丝并不随时间旋转)。观测到的由螺旋电流丝引起的三维边界磁拓扑改变和磁力线迹线程序所做的估计之间进行了对比。

EAST (大半径和小半径分别是 \SI{1.85}{\metre} 和 \SI{0.45}{\metre}) 是为了实验稳定的长脉冲、高参数的 H-mode 等离子体而建造的装置,它的位型与加热设备 ITER 类似,即有着灵活可调的 double null, lower single null (SN) 或 upper (SN) 极向偏滤器位型并主要是射频加热。EAST 中的低杂波系统工作在 \SI{2.45}{\giga\hertz},一个阵列由 20 个波导天线组成,四列五行,安装在低场侧中间,最大输出功率是 \SI{2}{\mega\watt}。其最初被设计用于芯部等离子体电流驱动,通过将电子朗道阻尼将动量转移给等离子体。峰值平行方向波折射率约 2.1。并且这套低杂波系统可以在没有离子回旋共振加热(ion cyclotron resonance heating, ICRH)的条件下仍实现长脉冲下的\Hmode 。然而和其他设备上的实验类似,显著的低杂波功率会损失在等离子体边界上面,特别是当等离子体密度较高时,这是由快波和粒子之间复杂的耦合问题所导致的。

低杂波对于边界局域模的特性影响,通过在离子回旋共振加热占主导的\Hmode 等离子体中调制低杂波功率中进行了研究。在这项实验中,目标 \Hmode 等于几?有一个,下SM配置这个高\Hmode 等议题主要是由你只会控制驱动的,他的输入功率是 \SI{1}{\mega\watt},在一个相对高密度的区域啊,在礼毕包成之后,problem。在边界上的安全系数是3.8。同时还有一个还相等于是电流在500千左右,以及还现场,1.8特斯拉,底部的三角系数是0.45。在等式中中心线密度的平均密度10是4.7×10的19次幂\SI{4.7}{},每立方米,以及,Greenwald 系数约 0.9,H 系数(H98y)在 \Hmode 阶段约 0.8。

这套低杂波系统功率设置在 \SI{1.3}{\mega\watt}。调制频率为 \SI{10}{\hertz}。运转时长占周期比例 50\%,于是,低杂波关闭的一个项相位时间是\SI{50}{\milli\second},这个时间大概是能量约束时间的一半,如果没有低杂波系统的话,边界局域模的频率是非常规律的,大概在 \SI{150}{\hertz}左右,当低杂波系统打开之后,人家就用膜消失了,或者problem,很奇怪的,出现在一个非常高的频率,大概在600号之后只左右,如图一所见。边界局域模,风之利刃,流强出现了显著的下降。大概有一个因子,2,并且有一个显著的增强在边界局域模之间的粒子数强,也是系数大约是。但仍然低于在l模工作室。这一现象在p7平板上面观测到了,在使用低杂波的时候,。等于其所具有的能量,大概有一个,因此为2的涨幅,从五十千焦到一百千焦,一旦\Hmode 。一旦\Hmode 成功啊。并且在低杂波功率调制期间,它的变化幅度比较微弱,大概在$\pm 5\%$区间之内。只有两个天然气平板所接收到的女子刘蔷,一旦关闭了低杂波系统,就会有一个迅速的衰弱,这可能揭示着低杂波功率,不仅仅是在本心不被吸收,同时他也,沉积在了刮削层中。

EAST 实验中,不管是\Lmode 还是\Hmode ,低杂波运行期间刮削层中都观测到了 5 条螺旋辐射带(Helical radiation belt, HRB)。螺旋辐射带的数量和低杂波天线阵列的行数是一样的。EAST 使用氦气放电使其结构更清楚而不改变其总的特性。作为一个典型例子,图 2 表示了两个切向方向上可见光波段的照片,这是氦气等离子体放电过程中,从 EAST 环形腔两侧 低杂波应用过程中所出现的图片。



这艺术目标等于几,300千安,磁感应强度两个特斯拉,$B=\SI{2}{\tesla}$, $q=0$安全系数为8左右,试油等,是由低杂波加热,大概功率有 \SI{0.7}{\mega\watt},他的拓普卫星是一个 double null 的配置。再等一日起,分割面以及,外层中央平面显示器之间的间隔大概在 \SI{8}{\centi\metre}。螺旋辐射带,沿着刮削层流动,在低场次,他同时向上面和下面的偏滤器,沿着磁场线同时流动,在天线前面。p刮削层中的磁力线,大概在等于己边界一厘米之外,就在低杂波天线前面。实验和模拟计算结果在位置和 pitch 角度上面都有很好的吻合。

% FIXME pitch angle

在等离子体边界螺旋丝状结构中流动的电流所激发出来的磁扰动,已经通过 Mirnov 线圈观测到了,发生在低杂波系统调试过程中。在这项实验中低杂波功率,用方波对它进行调整,功率周期性在\SIrange{0}{1.2}{\mega\watt}之间转换,它的重复频率是\SI{100}{\hertz},运转时长占周期 50\%。螺旋电流丝的产生是相当快速的,约\SI{2}{\milli\second}之内,这么短的时间对应着低杂波系统的启动时间。这就是系统开始的时候,螺旋电流丝,电流好散,在几个毫秒之内,电流丝所启动的拓朴结构改变,是在极向和环向上都是对称的,它表明了。等离子体结构的三维扭曲,。总的螺旋电流丝,电流强度。可以通过乔昌江模拟中,计算的扰动场强度与实验中的进行配合,被发现它的强度在。左右,在这项实验中。

在偏滤器上落点(Strike points, SP)的分裂,在低杂波加热系统运转的时候被观测到了。其分裂落点的效果和共轭磁扰动相似,可以在偏滤器平板上的热负荷分布上所观察到。图 4 显示了通过一个红外摄像头测量的外侧下部偏滤器平板的表面温度。平板上面,热负荷。。参照图片是。在低杂波使用过程中,外侧低处的天然气平板上面,表面温度,他表现出一个和原有的打击点,所截然不同的多分裂模式。间的距离,取决于环向角,这表明低杂波造成的磁拓扑是三维结构。另外落点分裂还取决于边界的安全系数,这一点,欧姆加热主导的和离子回旋加热主导的等离子体中都没有发现。
% FIXME Connection Length
% FIXME equilibrium 
% FIXME lobe
通过在磁力场线追踪的程序中考虑螺旋电流丝,如图 5 ,本文定性地模拟了磁拓扑结构的改变。在实验平衡态磁场基础上,加上从螺旋电流丝来的磁扰动场,电流丝测量到的电流强度为 \SI{1.3}{\kilo\ampere},以此来计算磁力线的重联长度。螺旋电流丝产生的磁场,在 X 点附近形成了数瓣有着较长重联长度的磁力线,直达外侧偏滤器平板造成打击点的分裂,使得落点发生了分裂,这一过程被红外摄像头所识别。计算结果表明,等离子体边界的剧烈变化,取决于边界的安全系数,和流经电流丝的电流强度。还要注意该螺旋电流丝模型没有考虑进去等离子体反馈,并且模拟的结果,只能定性地解释,螺旋电流丝激引起的落点分裂。
% FIXME small ELM
过去的的实验结果已经表明低杂波可以引起震荡型的边界局域模\Hmode 在 SN 偏滤器位型上(JET),也可以在限制器位型下产生没有边界局域模的\Hmode 等离子体(JT60),然而,它具体的物理机制还没有被充分研究。在 EAST 上,对离子回旋共振加热占主导的较低密度等离子体($n_e/n^{GW}<0.5$,这里 $n^{GW}$ 为 Greenwald 密度极限),用恒定的低杂波功率可以得到一个边界局域模较平稳的\Hmode 等离子体,其边界局域模有着混合的 type-I 和小类型的。通过降低离子回旋共振加热对低杂波加热的比例,再提高等离子体密度,达到了一个持续发生小类型边界局域模的\Hmode 等离子体,并且保持了32秒。低杂波激发的螺旋电流丝及其造成的磁拓扑改变迹象合理地解释了为什么低杂波可以减弱或者抑制边界局域模,并且显著地改变对偏滤器平板上的热负荷分布。对于这种现象背后物理机制的理解,需要考虑这种抑制效果关于以下几个因素的依赖关系,(i) 锂壁涂层,(ii) 等离子体碰撞,(iii) 安全系数,它们将会在未来 EAST 上的实验进行进一步的研究。

由于低杂波天线几何因素的影响,由螺旋电流丝所驱动的共轭磁扰动场主要是 n=1 的分量,在这里 n 指环向模数。基于实验参数计算出来的磁扰动场的谱表明螺旋电流丝的次扰动场共轭能力较好,从图 6 中可以看到,等离子体边界共振面和共轭磁扰动途中的脊线相贴合的。另外,由螺旋电流丝所引起的磁扰动,更多地位于等离子体边界,没有对核心部分显著的影响,这主要是由于螺旋电流丝在等离子体边界的刮削层中沿着磁力线进行流动。因此,螺旋电流丝的迹线总是紧密地贴合着边界磁力线的 pitch,而与边界安全系数无关。

还需要提到,尽管低杂波在刮削层中引起电流的现象已经被许多设备上观察到了,然而它的物理机制依然不清楚,在 Alcator C-Mod 的实验装置上面,当我们将低杂波注入方向改变的时候,刮削层中的电流方向并不会发生改变。Alcator C-Mod 上,在低杂波功率约 \SI{850}{\kilo\watt} 时,若等离子体运转在高密度状态,在刮削层中估计电流强度可以高达约 \SI{20}{\kilo\ampere},而 EAST 上低杂波功率为 \SI{1.3}{\mega\watt} 时,极向上积分得到的螺旋电流丝则约\SI{7}{\kilo\ampere}。
用 GENRAY-CQL3D 程序对考虑碰撞阻尼的二维刮削层模型模拟表明,EAST 上运行的高密度等离子体, 大概 10\% 的低杂波功率沉积在了刮削层中。实验观测到的结果,表明刮削层中的电流过大以至于不能通过直接的电流驱动,,引起对低杂波的碰撞吸收,来解释,不过要注意,刮削层中的低杂波功率吸收会对偏滤器区域中性粒子的电离有所贡献,从而增强了,沿着刮削层中的磁力线从较热较稀疏的偏滤器平板到较冷较稠密的偏滤器平板的热电电流,。
% thermoelectric current
EAST 过去的研究表明,随着低杂波功率或者等离子体密度的增长,螺旋电流丝电流强度均会增长。然而,螺旋电流丝所处的径向位置在刮削层中的分割面附近,而此处重联的磁场线长度远远大于电子的平均自由程。为了用低杂波实现对边界局域模和偏滤器平板上热负荷主动的控制,螺旋丝中电流强度对实验参数的依赖关系将会在 EAST 上面更进一步地被研究。

总的来说,低杂波对于边界局域模强烈的影响已经在 EAST 上面得到了呈现,它表明边界局域模在其作用下会消失,或者偶尔出现,它的频率会从 ~150 增加到 ~\SI{600}{\hertz},当低杂波进行驱动的时候,低杂波似乎通过驱动沿着刮削层磁力线且不随时间环向旋转的螺旋电流丝,来引起磁拓扑上显著的改变。这导致了在偏滤器上的落点分裂,与共轭磁扰动引起的效果相仿。在磁力线追踪程序中引入螺旋电流丝能较好地复现出来磁拓扑所观测到的改变。这为下一代的聚变反应堆(ITER 或 DEMO)提供了一种很有吸引力的手段来优化热负荷分布,并且同时抑制或者削弱边界局域模导致的极大的瞬态热负荷和粒子流强。

这项研究由中国国家磁约束聚变科学项目支持,项目序号为 No. 2013GB106003 和 No. 2011GB107001。在这里还要致谢德国亥姆霍兹协会的亥姆霍兹大学青年研究者团体  VH-NG-410。



\title{EAST 托卡马克上共振磁扰动引起的边界局域模从被削弱到完全抑制的非线性转换}

{\heiti 摘要:} 本文呈现了 EAST 托卡马克上如何用共轭磁扰动使得边界局域模从被减弱到抑制的非线性转换。这是第一次对射频加热占主导且缓慢旋转的等离子体的边界局域模用共轭磁扰动的方法进行抑制。在转化发生之后,边界磁拓扑的改变有两个迹象,线性磁流体动力学和真空的模拟结果等离子体反馈场渐变的相移和骤然的射向偏滤器的三维粒子束流。转换的阈值依赖于共轭磁扰动场的磁谱、等离子体自身的旋转及扰动场的幅度。这表明非线性等离子体反馈引起的边界磁拓扑结构改变在用共轭磁扰动的手段抑制边界局域模时很重要。


无论是在实验室等离子体物理还是在空间等离子体物理研究中,磁场重联及其导致的拓扑变化在等离子体动力学中都扮演了一个重要角色。通过共振磁扰动所引起的边界随机场,被认为是抑制等离子体边界周期性破裂发生的手段;该破裂也被称为边界局域模,起初在 DIII-D 托卡马克中被观察到。边界局域模会对直面等离子体的材料形成瞬态热负荷,并可使得它们性能下降,工作在下一代的聚变设备中(如 ITER)。在边界压力梯度和电流中储存的自由能,由于边界随机场的存在而减少,随机场将等离子体引入一个相对于边界局域模更稳定的状态。DIII-D 上成功的实验,激励了其他托卡马克设备运用共轭磁扰动控制边界局域模。然而等离子体反馈场往往会屏蔽施加的共轭磁扰动,并且可能显著降低磁场的随机性,这一机制能否成功应用还需研究。与拓扑结构改变不同,线性的 peeling like 磁流体动力学反馈,已经被发现在边界局域模控制中扮演者很重要的角。非线性的等离子体响应,已经在 JET 托卡马克中被观测到了。近期,DIII-D 上发现了在边界局域模抑制阶段,施加 $n=2$ 共轭磁扰动场而产生边界磁岛可能的形成机制。然而,在边界局域模被完全抑制和削弱之间的关键性区别仍不清楚,以及线性和非线性的等离子体反馈在边界局域模抑制上的作用仍有待研究。

这篇快报阐述了第一次对缓慢旋转且射频加热主导的的等离子体,以低 $n$ 的共轭磁扰动场驱动边界局域模抑制的效果,这可能对于未来聚变设备有着重要价值。这也是第一次 EAST 工作在中等碰撞率状态时观测到以共轭磁扰动达到完全边界局域模抑制的效果,同时拓展了过去边界局域模抑制在 DIII-D 和 KSTAR 的观测结果。目前发现,边界附近的磁岛和超过阈值的磁拓扑改变(考虑等离子体反馈的条件下),都在边界局域模抑制中起着重要作用,这一发现也揭示了,在线性和非线性反馈中在边界局域模抑制中的不同作用。

2014 年在 EAST 低场侧安装了两个阵列组成($2×8=16$)的一套灵活的腔内共轭场线圈系统。EAST 团队通过 $n=1,2$ 的共轭磁扰动,成功地实现了对缓慢旋转且射频加热主导的的等离子体中 type-I 边界局域模的减弱及完全抑制。

EAST 中观测到,$n=1$ 的共轭磁扰动场,强度超过了阈值时具有对纯射频加热的等离子体边界局域模彻底抑制的效果。图 1 显示 EAST 实验序号 55274 中,$n=1$ 共轭扰动场线圈电流慢上升过程中边界局域模的表现。低杂波电流驱动 $P_{LHCD}=\SI{3}{\mega\watt}$,以及离子回旋共振加热 $P_{ICRF}=\SI{1.4}{\mega\watt}$ 提供恒定的外部加热功率。 X 射线晶体成像技术(XCIS)测量得到环向上绕等离子体中心的旋转速度非常接近于0,(<\SI{4}{\kilo\rad\per\second})。环向磁感应强度为 $B_T = \SI{2.25}{\tesla}$,在表面 $95\%$ 的归一化极向磁通处的安全因子 $q_{95}\approx 5.7$,等离子体电流等于 $I_p=\SI{0.45}{\mega\ampere}$,归一化贝塔参数 $\beta_N \approx 0.8$,以及归一化的碰撞率在平台顶部是差不多,约等于1。如图 1 所示,共轭磁扰动场升到 \SI{8}{\kilo\ampere turns} 之前(\SI{6}{\second}),电子密度呈阶梯状下降并且边界局域模发生频率,而在 $t=\SI{6}{\second}$ 在此之后边界局域模被完全的抑制住了。
% FIXME turn 转
真空模拟中磁岛重叠处的宽度,如图 1,$\Delta_{\Sigma>1}=1-\hat{\psi}_p^{1/2}|_{\sigma=1}$(黑实线),此处 $\sigma$ 为 Chirikov 参数表征磁岛的重叠条件而 $\hat{\psi}_p$ 指归一化之后的极向磁通。等离子体反馈在边界局域模的削弱和抑制阶段有着显著的不同。耦合等离子体反馈后 $n=1$ 磁扰动场的幅度(红实线)在实验中观测到的值可见图 1(b)。
在完全抑制之前电子密度阶梯状的下降趋势和边界局域模频率的增长,其原因可能是由于不同谐波分量有着不同的渗透阈值,说明(考虑等离子体反馈后的)磁拓扑结构改变的程度在最终的边界局域模抑制中非常重要。这激励团队对边界局域模在被削弱和抑制之间的转化过程进行细致研究。


通过扫描托卡马克上下沿磁传感器的相位差 $\delta\Phi_{UL}$,或者是等效的共轭分量场强,都可以说明边界局域模减弱和抑制之间的转化。
图 2 展示了边界局域模控制机制,我们有一个连续性的扫描,相差[图 (b) 中红线]。通过旋转下面的线圈电流,再一个,$f=\SI{0.5}{\hertz}$ 的频率,并且保持上沿线圈电流恒定 $I_{RMP}= \SI{10}{\kilo\ampere turns}$ 实验序号 55272。该目标等离子体边界局域模的频率大致为 \SI{100}{\hertz} 与实验序号 55274 类似,除了加热手段上的略微不同,他有着 \SI{0.7}{\mega\watt} 的反向中性束注入而不是离子回旋共振加热。它仍然是射频加热主导的等离子体。电子密度[图 2(c) 中的实线]和边界局域模频率[图 2(b) 三角形]的改变。和,可重复性比较好,在两个阶段。它有着明显的33各项阶段图 4(a) 和 4(b) 也说明了这一点。整个过程有明显的三个阶段。在第一个阶段。强烈的密度和边界局域模削弱,但是边界局域模的频率增强了,因此大概在5~10左右膜再向阶段二中完全的抑制住了,在一个突然的电子密度下落之后,在第三阶段,相位差的其余部分。相当相当弱,并且总是保持着一个阐述在一个突然的转换中,边界局域模抑制中,拖出来,电子密度和温度。形貌的变化,在图中有所展示。从中我们可以看到电子密度下降了,但温度上升了,在工厂缠绕中的应用阶段,等离子体能量约束,在边界局域模的意义削弱中,是似乎有一些轻微的便好,更多的储存能量,当然更低的密度。比赛共轭磁扰动,应用之前要好,和边界局域模的抑制相削弱相比,完全的抑制,有一个更强,强得多的密度潘博奥效应,和一个轻微的。边界平台温度下降,从边界局域模的削弱到完全抑制,能量约束下降了大约10\%。

边界局域模控制对磁谱的依赖性表明,共振磁扰动场需要达到必要的阈值才能抑制边界局域模。和在 DIII-D 中观测到的类似,边界局域模抑制最合适的相位差,和线性磁流体模拟程序 MARS-F 得到的共振峰值接近($\approx 75$ \degree),而与真空条件 MARS-F 计算出来的共振峰值计算($\approx 356$\degree)不吻合。然而,EAST 上等离子体密度的时间演化和 DIII-D 上完全不同,DIII-D 上在 $\delta\Phi_{UL}$ 相位差扫描时等离子体密度抽出?并且磁刹车如同三角函数一个变化,显示出等离子体动理学特性的线性反馈。
% FIXME density pump-out and magnetic braking 
% equilibrium 平衡磁场
% TODO
实验测量结果清晰地表明边界局域模减弱和抑制之间的转换,存在着非线性等离子体反馈的作用以及非线性转换和分歧过程,如图 3 和 4 所示。EAST 团队运用低场侧遍布环向各角度的极向磁场传感器(如图 3 所示)来观测等离子体反馈场的演化是。通过反馈场的傅立叶分解得出其主要的分量是 $n=1$ 的谐频分量。MARS-F 程序从等离子体反馈场中模拟出扰动场,在图 3(c) 中可以看出其较好地重现了总体的趋势,只有些许的不一样。实验序号 55272 在 $\SI{3.9}{\second}$ 的平衡态被用于这里展示的模拟,这是因为等离子体反馈场的预估,并不因有或者没有共轭磁扰动导致的边界局域模抑制的磁场平衡态而产生显著的差异。
然而,$n=1$ 反馈场的相位对 $\delta\Phi_{UL}$ 的依赖关系(如图 4(c))明确地表征了边界局域模减弱和抑制之间的非线性特性。
弱边界局域模减弱阶段(III),$\delta\Phi_{UL}\in [120,360]$\degree, 测量到的 $n=1$ 反馈场和线性磁流体反馈相吻合,而在边界局域模抑制阶段(II)$\delta\Phi_{UL}\in [50,120]$\degree,它显著地偏离了线性的磁流体反馈却跟真空的更符合。
这表明共振磁扰动在边界局域模弱化状态被等离子体屏蔽掉了,到抑制状态却得以渗透进去。这是因为渗透的共轭分量和真空中的一样有相同的相位;而根据过去的非线性模拟,屏蔽场相对于真空有一个相位偏移。这意味着,为了达到边界局域模抑制状态,磁场渗透的发生是必需的,而这不能通过线性的模拟解释。这有可能能够解释 DIII-D 中测量得到的反馈场和 MARS-F 模拟结果的类似的差异。
但和 DIII-D 中观测到边界局域模抑制时的磁场渗透特性不一样的是,EAST 上渗透的环向模数和施加的是一样的。



在边界局域模被减弱到完全抑制的转换过程中,反馈场的相位逐渐地和真空中的不断逼近,就图 4(c) 所示。这表明不同的谐频分量依次穿透,边界拓扑变化在这一阶段渐渐剧烈起来。对于不同的谐频分量来说渗透阈值可能是不一样的。于是一种可能的解释是这个转化过程中有多种谐频分量依次穿透。这也解释了在完全抑制边界局域模之前,共轭磁扰动线圈电流的上升时,电子密度和边界局域模频率的阶梯状变化的现象,如图 1。于是,边界拓扑改变的剧烈程度随着总共轭磁扰动场幅度增强而增强,这包括了等离子体反馈和它导致的边界局域模抑制。从抑制边界局域模(阶段 II)到轻微减弱边界局域模(阶段 III)的骤然反向转换表明这些共轭谐频分量又几乎同时被屏蔽掉了;同时,磁场的三维结构消失了,共轭磁扰动场强度低于某个阈值之后。
% strong ELM mitigation ? 强的边界局域模?边界局域模的强抑制?
在边界局域模抑制阶段,粒子流受共轭磁扰动影响而在偏滤器上落点的分裂,和边界拓扑结构的变化相互佐证。在 DIII-D 上的\Lmode 等离子体,只有在等离子体屏蔽效应退去的时候,才能观测到三维的落点分布。当共轭磁扰动场强度超过阈值时,MAST 上\Lmode 等离子体出现粒子流和热负荷在偏滤器上的三维分布时,总是伴随着突然的热流增强和等离子体密度减弱,这表明存在边界随机场。
这已经在l模等离子体中被观测到了在max上面。这种分裂模式在强的边界局域模削弱阶段和抑制阶段,被观测到时用一个极向排列的兰缪探针列,在上偏滤器,在 $\Phi =327$\degree 的时候,观测到图3,这和真空的三维模拟,打几点去的?结果是相吻合的,显著的粒子流强增加,也表明了俺的穿透和磁拓扑结构改变,在这些阶段,因为这是下 SN 位型,其中的两个分界面之间的距离 $d_{rsep}\approx \SI{1}{\centi\metre}$。这再一次证明了边界局域模抑制阶段存在边界磁拓扑的改变。
% 边界垂直转动 edge perpendicular rotation
从边界局域模削弱到完全抑制时,会突兀地加快边界垂直转动,再一次佐证边界局域模完全抑制时边界拓扑改变的重要性。外沿边界的加速旋转是边界随机场形成的重要迹象。图 5 显示了在两次实验 56365 和 56366 之间对边界局域模控制效果的对比,它们有着相同的共轭磁扰动场参数为行,和目标等离子体,$B_T=\SI{1.7}{\tesla}$,$I_p=\SI{0.45}{\mega\ampere}$, $\beta\approx 1.5$ 和 $q_{95}\approx 4.5$。除了 56365 号顺向中性束功率为 \SI{2}{\mega\watt}在 $t\approx \SI{3.6}{\second}$ 降到 \SI{1.2}{\mega\watt},就如图 5(b) 所示。图 5(a) 一个阶梯,旋转共轭此共轭磁扰动,有着这样子的类型,相位保持着恒定的电流 $I_{n=2}=\SI{14}{\kilo\ampere turns}$ 且 $\delta\Phi_{UL}=270$\degree,在这两次实验上都应用了。
这两次实验观测到了,在共轭磁扰动应用之后的边界局域模强烈减弱阶段,频率大概变为了原来的 5 倍。边界局域模的完全抑制,只有在一个额外的瞬间边界垂直旋转加速之后才会实现在一个缓慢的衰弱,由中性束功率的减低导致的等离子体旋转渐弱之后。 Mirnov 信号 $dB/dt$ 被用来作为测量边界局域模破裂的手段,因为它在强减弱阶段对小的边界局域模破裂更加敏感。如图 5(b) 所示,就在共轭磁扰动施加之后,多普勒反散射系统观测到了边界垂直旋转的骤然加速,这表明边界拓扑结构的改变。而图 5(f),5(g) 展示了共轭磁扰动启动后,因为旋转急停了,估计的电子流体垂直旋转角速度 $\omega_{e\perp}$,及$\Vect{E}\times\Vect{B}$ 对应的 $\omega_{E\times B}$ 在基座顶部俊变得非常接近于 0。根据近期的等离子体反馈理论和模拟结果,在基座顶部附近(此处 $\omega_{e\perp}\approx 0$)的共轭谐频分量可能渗透。在边界局域模抑制时,基座顶部处 $\omega_{e\perp}$ 又更接近于 0,此处 $\rho \approx 0.9$,而且$\omega_{E\times B}$ 在 $\rho=0.92$ 以内分布相当平坦。
更多的模式可能会渗透进等离子体,促成了最终的边界局域模抑制效果。以上论述表明强烈的边界局域模减弱效果和磁渗透及边界拓扑结构的改变是相关联的,并且最终到边界局域模完全抑制阶段的转换需要边界拓扑变化达到一定剧烈程度。

% pedestal top

在此总结, EAST 托卡马克中观测到了用 $n=1,2$ 的共轭磁扰动,驱动边界局域模削弱到完全抑制的非线性转化的迹象。在低速旋转且射频加热为主导的等离子体中,低 n 的共轭磁扰动场抑制边界局域模的作用,对于未来的聚变设备中可能有潜在的重要价值。线性的磁流体模拟结果揭示了总的共轭磁扰动场强度,它考虑了等离子体反馈场,而不是真空的那个,从而优化设计了适合抑制边界局域模的线圈卫星,对于完全的边界局域模抑制,反馈场的相位逐渐的偏离了现实的结果,并且接近真空的场,从边界局域模削弱到抑制的过程中。
这表明不同的谐频分量穿透依次穿透。并且边界拓扑,改变的程度不断加深在这个过程中,这也解释了,被观测到的阶梯状的改变,电子密度阶梯状的改变和边界局域模频率的改变,在共轭吃少动,启动的过程中,在边界局模抑制之前,打击点的分裂和突然的粒子增长粒子刘翔增长,粒子刘翔打到偏滤器上的增长也支持着边界拓扑改变在边界局域模抑制阶段。边界垂直旋转触发,边界垂直旋转的急剧增加触发了,从边界局域模削弱到抑制的转化,并且这也表明,存在着一个玉字,在边界拓扑结构改变的过程中来达到完全抑制。然而非线性等离子响应对于各个词共轭磁扰动的响应的模拟仍然是一个巨大的挑战,在未来的研究中,更多的努力将会被投入到理解,非线性等离子体响应特别在于对于转化和分差的关键问题。

这项工作受到中国国家自然科学基金会的支持,项目序号为 No. 11475224 和 No. 11205199;同时还受到中国国家磁约束聚变科学计划支持,项目序号为 No. 2013 GB102000, No. 2013 GB106003B 和 No. 2012 GB105000.

% \chapter{其它附录}
% 前面两个附录主要是给本科生做例子。其它附录的内容可以放到这里,当然如果你愿意,可
% 以把这部分也放到独立的文件中,然后将其 \cs{input} 到主文件中。