\newcommand{\eqvp}{\hyperref[eq:vp]{(VP)}~}
\newcommand{\eqvm}{\hyperref[eq:vm]{(VM)}~}
\newcommand{\eqrvp}{\hyperref[eq:vp]{(RVP)}~}                      
\newcommand{\eqrvm}{\hyperref[eq:vm]{(RVM)}~}

\chapter{介绍}
% \chapter{Introduction}



\section{研究背景}
% \section{Background}

Vlasov 类型的偏微分方程系统是描述粒子群多体运动问题的 Boltzmann 方程在无碰撞条件下的简化。粒子群在给定的 $(t,\vx)$ 处的各向异性的速度分布对系统变化产生了很大的影响,使得对粒子速度空间分布的刻画十分必要。通过将速度空间分布考虑到系统中,即分布函数从时空分布的变为更细致的相空间分布,Vlasov 类型的偏微分方程系统从而能够精确地描述动理学意义上的运动演化规律。准确来说,相空间的分布函数 $\ftxv ~ (x\in \bbRx,~v\in\bbRv,~t\geq 0) $。当电磁力作为主要考虑对象时,即与 Maxwell 方程组耦合的时候 \eqvm,Vlasov 方程描述的便是带电粒子与电磁场相互作用的关系,从而描绘物质电磁相互作用的图象。
% The Vlasov type equation is studied as a governing equation describing the multi-body motion of the particle swarm, 
% in which the anistropic velocity distribution contributes a significant influence to the dynamics of the system. 
% Phase space distribution $\ftxv\geq 0 ~ (x\in \bbRx,~v\in\bbRv,~t\geq 0) $ with initial data $f_{0}(\vx, \vv)=f(0, \vx, \vv)$ determines the particle density at $(t, \vx,\vv)$, \textit{i.e.}, the number of particles per unit volume in phase space. When coupled with Maxwell equations as the electromagnetic field governing rule, Vlasov equation is capable to decide the dynamic 
% scenario for particle-field interaction, named Vlasov-Maxwell system \eqvm.


% The Vlasov type equation is studied as a governing equation describing the multi-body motion of the particle swarm, 
% in which the anistropic velocity distribution contributes a significant influence to the dynamics of the system. 
% Phase space distribution $\ftxv\geq 0 ~ (x\in \bbRx,~v\in\bbRv,~t\geq 0) $ with initial data $f_{0}(\vx, \vv)=f(0, \vx, \vv)$ determines the particle density at $(t, \vx,\vv)$, \textit{i.e.}, the number of particles per unit volume in phase space. When coupled with Maxwell equations as the electromagnetic field governing rule, Vlasov equation is capable to decide the dynamic 
% scenario for particle-field interaction, named Vlasov-Maxwell system \eqvm.


\begin{equation}\label{eq:vm}\text { (VM \& RVM) }\left\{\begin{array}{l}
    f_{t}+\vect{a}(\vv) \cdot \nabla_{x} f+(\vE+\vect{a}(\vv) \times \vB) \cdot \nabla_{v} f=0 \\
    \vE_{t}=\nabla \times \vB-\vect{j} \\
    \vB_{t}=-\nabla \times \vE \\
    \nabla \cdot \vE=\rho, \quad \nabla \cdot \vB=0
    \end{array}\right.\end{equation}
其中 $\vect{a}(\vv) \in\{\vv, \hat{\vv}\},~ \hat{\vv}:=\vv / \sqrt{1+|\vv|^{2}}$ 且 
\begin{equation}
\rho(t, \vx):=\int_{\bbRv} \ftxv d \vect{v}, \quad \vect{j}(t, \vx):=\int_{\bbRv} \vect{a}(\vv)\ftxv d \vect{v}.
\end{equation}
此处 $\vE, \vB, \rho$ 和 $\vect{j}$ 各表示电场、磁场、空间密度分布和电流密度分布。相对论提出的时空理论,认为光速是有限的,且所有物质的速度都慢于光速 $\vect{a}(\vv)=\hat{\vv}< 1$,从而有非相对论型的 Vlasov-Maxwell 系统 \eqvm 和非相对论型的 Vlasov-Maxwell 系统 \eqrvm。
% Here $\vE, \vB, \rho$ and $\vect{j}$ expressed electric field, magnetic field, spatial density and current density respectively. The more dedicated Einstein theory considers relativistic effect when $\vect{a}(\vv)=\hat{\vv}$ and limit the maximum velocity the particles can reach, transforming the Vlasov-Poisson to relativistic Vlasov-Poisson \eqrvp ~and the Vlasov-Maxwell to relativistic Vlasov-Maxwell \eqrvm.

% One could easily capture the physical essential of the Vlasov type equaiton by observing the $\mu(\vE+\vect{a}(\vv) \times \vB)$ term which is indeed the acceleration of particles located at $(\vx,\vv)$ in the phase space.


进一步简化,当电磁相互作用中静电相互作用力占主导时,洛伦兹力、磁场等的要素可以被简化掉,从而我们可以电磁学中的 Vlasov-Poisson 系统。同时,Vlasov-Poisson 系统还在天文学中也起到重要作用,它们有以下的一般形式,区别在于我们加入了 $\mu$ 将 $\vE$ 的方向置反。
% Furthermore, under the assumption that the electrostatic force dominates the interaction, \textit{i.e.} the Lorentz force could be treated as zero, magnetic force is omitted and then comes the Vlasov-Poisson system \eqvp. $\vE$ could be expressed as the gradient of the electrostatic potential under the assumption.
\begin{equation}
    \label{eq:vp}
    \text { (VP \& RVP) }\left\{\begin{array}{l}
    \partial_{t} f+\vect{a}(\vv ) \cdot \nabla_{x} f+\mu \nabla_{x} \phi \cdot \nabla_{v} f=0 \\
    \Delta \phi=\rho(f):=\int_{\bbRv} \ftxv d \vv
\end{array}\right.\end{equation}
其中 $\mu \in\{+,-\}, \vect{a}(\vv) \in\{\vv, \hat{\vv}\},$ 且 $\hat{\vv}:=\vv / \sqrt{1+|\vv|^{2}}$. 
% where, $\mu \in\{+,-\}, \vect{a}(\vv) \in\{\vv, \hat{\vv}\},$ and $\hat{\vv}:=\vv / \sqrt{1+|\vv|^{2}}$. 
$\mu$ 的符号正负表示不同的物理图象,其为 "+" 表示等离子体物理中同种电荷的相互排斥的库伦作用,而 "-" 表示星体动力学在万有引力主导下的作用规律。简化后的 \eqvp 系统对应与 \eqvm 系统 \ref{eq:vm} 有如下的特殊电磁场,
\begin{equation}
    \vE(t, \vx)=-\nabla_{x} \phi(t, \vx), \quad \vect{B}=0, \quad \phi= \frac{1}{|\vx|} *\rho 
\end{equation}

\eqvp 实际上表明了粒子群在一种势场作用下的运动规律,这种势场由粒子本身产生,在 $N=3$ 维空间中场强与 $1/r^{2}$ ($r$ 为粒子之间的距离) 成正比,这使得 Vlasov-Poisson 问题中的场可能 $\vE \notin \mathrm{L}^{\frac{3}{2}}$。%当三维情况时,这意味着 $\vE \notin \mathrm{L}^{\frac{3}{2}}$。
当 Vlasov-Poisson 系统用来讨论近似静电学问题时它表明的是排斥的库伦相互作用 $(\mu=+1)$,而在星体动力学中则是万有引力的相互作用($\mu = -1$)。
% in fact indicates the rules of swarn motion governed by the scalar potential field-particle interaction, in which that the scalar potential field is generated by the particles themselves and that the stellar dynamics (plasma physics) case shows the absorbing gravitation (the repulsive Coulomb force) respectively. 
% The sign of $\mu$ indicates different physical scenario, while "+" for the plasma physics case and "-" for the stellar dynamics case. \eqvp in fact indicates the rules of swarn motion governed by the scalar potential field-particle interaction, in which that the scalar potential field is generated by the particles themselves and that the stellar dynamics (plasma physics) case shows the absorbing gravitation (the repulsive Coulomb force) respectively. 

当考虑多粒子 (Multi-species) 相互作用问题时,和单粒子情况在数学上没有本质的区别,通过对不同种粒子给定其质量 $q_i$ 和电荷量 $m_i$ 即可求解,其方程不在此赘述。但注意在引力作用 $\mu=-1$ 时,没有多粒子的物理图像。 
% Multi-species extended equation of \eqvp and \eqrvp could be easily established by adding $q_i$ and $m_i$ in the original ones. Note that $\mu$ must be "+" at the present because the stellar dynamics case only allow single species situation. 

% \begin{equation}
%     \label{eq:mvp}
%     \text { (Multi-species VP \& RVP) }\left\{\begin{array}{l}
%     \partial_{t} f_i+\vect{a}(\vv ) \cdot \nabla_{x} f_i+ q_i/m_i \nabla_{x} \phi \cdot \nabla_{v} f_i=0, \\
%     \Delta \phi=\rho(f):=\int_{\bbRv} \sum_i q_i f_i d \vv
% \end{array}\right.\end{equation}

% Though \eqvm~ results give many heuristic clues in the research of \eqvp and much more is known about \eqvp. In this paper we mainly investigate the characteristics of \eqvp and \eqrvp.



\section{特征线}
% \section{Characteristics}
在 Vlasov 型问题的研究中经常使用的是偏微分方程中的常用技巧,特征线:$s \mapsto {X}(s, t, \vx,\vv),~ s \mapsto {V}(s, t, \vx,\vv)$,它定义为以下相应的常微分方程组的解:

% Frequently used in the context of Vlasov-problem research are the characteristics: $s \mapsto \vect{X}(s, t, \vx, \vv),~ s \mapsto  \vect{V}(s, t, \vx, \vv)$ defined as the solutions of the following corresponding system of ordinary differential equations:

\begin{equation}\begin{aligned}
    &\frac{d \vect{X}}{d s}=\vect{a}(\vect{V})=\{\vect{V}, 
        \vect{V} / \sqrt{1+|\vect{V}|^{2}}\}\\
    &\frac{d \vect{V}}{d s}=\mu \vE(\vect{X}, s)
\end{aligned}\end{equation}
具体来说,在 Vlasov 型问题中,它表示在 $t$ 时刻过 $(\vx,\vv)$ 点的特征线的轨迹,即 $ \vect {X} (t, t, \vx, \vv) = \vx, ~ \vect{V} (t, t, \vx, \vv) = \vv$。有时,当 $(t,\vx,\vv)$ 明确时,我们直接使用$\vect{X}(s)$来简化符号,特别是当我们研究单条特征线时。

沿特征线偏微分方程求解的函数值不变,因此,若初始数据有界,自然地有 $||f(t,\cdot,\cdot)||_\infty=||f_0||_\infty<\infty$。在下一章对 \text{VP} 奇性弱化后解的扩展中我们将讨论更多的特征线的性质。
% indicating the trace of a particle who arrives at $(\vx,\vv)$ at the time of $s=t$, \textit{i.e.}, $\vect{X}(t,t,\vx,\vv)=\vx, ~\vect{V}(t,t,\vx,\vv)=\vv$. Hence $||f(t,\cdot,\cdot)||_\infty=||f_0||_\infty<\infty$ by initial data boundedness assumption. Sometimes we use $\vect{X}(s)$ directly, when $(t,\vx,\vv)$ are known, to simplify the notation, especially when we are studying the traces of characteristics.


\section{适定性的相关结果}
% \section{Well-posedness Results}


我们将只研究 Vlasov-Poisson 问题的经典解,即在这些解上对应的特征线的常微分方程有着唯一经典解。这种情况下解的局部存在性对于给定的 $N$ 维空间已经由 \cite*{HorstClasssicalI} 证明,我们将在第二章做重点梳理。已知全局存在的情况如下:
% 本地解决方案的存在对于cite{HorstClasssicalI}中的mathbf{n}$中的每一个$n$ 都是已知的。


% 此外,我们知道对于$\left。
% n \geqslant 4 \text{有}f_0 \text{(甚至在}C_{0}^{infty}\left(\mathbf{R}^{n} \times \mathbf{R}^{n}\right),$,这样相应的解决方案只存在于有限的时间间隔[9]。

% We will study only classical solutions of (VP), i.e., solutions for which the characteristic system of (1.1) has unique classical solutions. In this case the existence of local solutions is known for every $n \in \mathbf{N}$ by \cite*{HorstClasssicalI}.  Global existence is known in the following cases:

非相对论情况:
\begin{enumerate}[(i)]
    \item $n=2$ , \cite*{Illner1979}
    \item $n=3$ 时 $f_0$ 球对称,\cite*{HorstClasssicalII}
    \item $n=3$ 时 $f_0$ 柱对称, \cite*{hellwig1964partial}
    \item $n=3$ ,\cite*{1991InMat.105..415L}
    \item $n=4$ 时 $f_0$ 球对称且足够小, \cite*{hellwig1964partial}
\end{enumerate}
% Non-relativistic Situation:
% \begin{enumerate}[(i)]
%     \item $n=2$ , \cite*{Illner1979}
%     \item $n=3$ and $f_0$ spherically symmetric, \cite*{HorstClasssicalII}
%     \item $n=3$ and $f_0$ cylindrically symmetric \cite*{hellwig1964partial}
%     \item $n=3$ , \cite*{1991InMat.105..415L}
%     \item $n=4$ and $f_0$ spherically symmetric and small, \cite*{hellwig1964partial}
% \end{enumerate}



相对论情况:
\begin{enumerate}[(i)]
    \item $n=3$, $\mu=1$ 且 $f_0$ 球对称; $\mu=-1$, 初值足够小且球对称, \cite*{glassey_symmetric_1985}。 两者初值都需要紧支集条件。
    \item $n=3$, $\mu=1$, 初值球对称且有局域约束条件, \cite*{wang2020global}. 
\end{enumerate}

% Relativistic Situation:
% \begin{enumerate}[(i)]
%     \item $n=3$, $\mu=1$ and $f_0$ spherically symmetric; $\mu=-1$, $f_0$ spherically symmetric and small \cite*{glassey_symmetric_1985}. Both with compact support assumption.
%     \item $n=3$, $\mu=1$, localized sphercial symmetric data, \cite*{wang2003global}. 
% \end{enumerate}



另外,$n \geqslant 4 $ 时,即使 $f_0 \in C_{c}^{\infty} \left(\mathbf{R}^{n} \times \mathbf{R}^{n}\right)$ 性质相当好,也不一定有全局解, \cite*{HorstClasssicalII}.
% Further, it is known that for $\left.n \geqslant 4 \text { there are } f_0 \text { (even in } C_{0}^{\infty}\left(\mathbf{R}^{n} \times \mathbf{R}^{n}\right)\right),$ so that the corresponding solutions exist only on a finite time interval [9].

% \footnote{The paper [16] listed in the reference of \cite*{1991InMat.105..415L} is missing, refered as Iordanskii, S.V.: The Cauehy problem for the kinetic equation of plasma. Transl., II. Ser.,
% Am. Math. Soc. 35, 351-363 (1964)}


对于经典解来说,其存在性和唯一性结果已经由 Iordanskii, \cite*{ukai1978classical} 和 \cite*{bardos1985global} 分别在一维、二维和三维下对小初值给出了。经典解的具有对称性的初值问题也有 \cite*{batt1977global}, \cite*{wollman-1980-symmetric}, \cite*{HorstClasssicalI}, \cite*{schaeffer1987global} 等讨论过,其中 \cite*{schaeffer1987global} 还在同一片文章中处理了相对论情况的对称初值问题.

% For classical solutions, it is well known that the existence and
% uniqueness result of the Vlasov-Poisson system solution have been presented by Iordanskii 
%  in dimension
% 1, \cite*{ukai1978classical} in dimension 2, \cite*{bardos1985global} in dimensions
% 3 for small data. 
% The case of (nearly) classical symmetric data has been treated by \cite*{batt1977global}, \cite*{wollman-1980-symmetric}, \cite*{HorstClasssicalI}, \cite*{schaeffer1987global}, among whom \cite*{schaeffer1987global} treated the relativistic case of symmetric data in one paper.

\textbf{特别的在三维空间的话},综述第二章讨论的局部适定性对 $N=3$ 没有给出全局的结果,弱解在相当弱的意义下全局存在 ( \cite*{arsenev_global_1975, abdallah_weak_1994} ), 如果 Cauchy 问题初值足够小全局经典解也可以存在, \citet*{bardos1985global}。
% In particular, in three space 
% dimensions, global weak solutions exist ( \cite*{arsenev_global_1975, abdallah_weak_1994} ), and global classical 
% solutions exist if the Cauchy data are small enough given by \cite*{bardos1985global}.



% 当我们考虑相对论情形时,\eqrvp 似乎看上去比经典的更好了,因为 $|\hat{\vv}| \leqslant 1$. 于是经典问题中高阶矩的发散困难便迎刃而解了。 
% 基于同样的原因,存在上限的速度导出了确定的因果关系(casuality),这些是好的方面。

% When we turn to the relativistic case, at first sight, \eqrvp seems "better" than its classical version, since $|\hat{\vv}|<1$. Thus 
% "higher moment difficulties" well-known in the classical case, will not occur. 
% Moreover, for the same reason we have the casuality along characteristics due to the limited velocity. These favorable circumstances are diminished somewhat by examination of the total energy integral. One may hope then that \eqrvm is better behaved than \eqvm, but only when $\gamma=+1$. 

当讨论到某些量的范数,\eqrvp 可能确实不如 \eqvp。比如 $\mu=+1$ 时,$\rho\in L^{4/3}(\bbRx)$ 是比非相对论的情形($\rho\in L^{5/3}(\bbRx)$)要更差的。 不过通过 \cite*{batt1977global} 和 \cite*{wollman-1980-symmetric} 
仍然可以证明当 $\mu=+1$ 时全局球对称解的存在性。而对于引力情况 $\mu=-1$,\cite*{glassey_symmetric_1985} 证明了其解的存在性被弱化了,只有在初值满足条件,足够小时 $40\mathcal{M}^{2/3}\|f_0\|_\infty^{1/3}<1$ 才能确保 \eqrvp 存在全局解。 同时其还举了不存在全局解的反例,此时若初始能量 $\mathcal{E}_0:=\int_{\bbRx}\left\langle\int_{\bbRv} \sqrt{1+|\vect{v}|^{2}} f d \vv+\frac{1}{2} \gamma|\vE|^{2}\right\rangle d \vect{x}$ 为负,那么这样一个球对称的经典解的不存在全局解,其延续时间必然有限。 
% In the plasma physics case, $\rho\in L^{4/3}(\bbRx)$ is worse than the result for (VP) itself, where 
% $\rho\in L^{5/3}(\bbRx)$. However, Batt's and Wollman's methods [\textit{cf.} \cite*{batt1977global}, \cite*{wollman-1980-symmetric}] 
% can still be adapted and used to show the existence of global spherically symmetric solutions when $\gamma=+1$. In contrast to the plasma case, \cite*{glassey_symmetric_1985} stated the existence of solution of stellar dynamics is weakened in the sense that a restrction on the size of initial data is necessary. Only "small" radial solutions with $40M^{2/3}\|f_0\|_\infty^{1/3}$ are confirmed to exist in the large for (RVP) with $\gamma=-1$. Indeed, if $\gamma =-1$ and the initial energy $\mathcal{E}_0$ is negative, it is shown that the life-span of such a radial, classical solution is 
% finite. 

% The outstanding result for the
% full 3D Maxwell-Vlasov system remains that of \cite*{glassey1986singularity}, who were able
% to prove a global existence result under the hypothesis of compactly supported
% (for all time !) particle density. 
% \begin{theorem}\textit{(Glassey-Strauss)}
%     \label{the:glassey-strauss}
%     For \eqrvm~ system, assume the standard regularity $f_0\in C_0^1$ and $\vE_0, \vB_0 \in C^2 $ for the initial data and assume there exists a continuous function $\beta(t)$ such that for all $\vx \in \mathbb{R}^{3}$
%     \begin{equation}
%       \label{eq:rvpglassey-straussbound}
%     f(t, \vx, \vv)=0 \quad \text { for } \quad|\vv|>\beta(t)
%   \end{equation}
%   Then there exists a unique $C^{1}$ solution of the system for all $t$.
  
%   \end{theorem}
  
%   The Glassey-Strauss proof relies on showing uniform bounds in time for the $||\vE||_\infty, ||\vB||_\infty$, $||f||_\infty$ as well as of all their first spatial derivatives. They started by rewriting \eqrvm~ as follows:
  
%   \begin{equation}\left\{\begin{array}{l}
%     \partial_{ t} f+\hat{\vv} \cdot \nabla_{x} f+(\vE+\hat{\vv} \times \vB) \cdot \nabla_{v} f=0, \quad(\vx, \vv) \in \mathbb{R}^{3} \times \mathbb{R}^{3} \\
%     \vE_{t t}-\Delta \vE=-\left(\partial_{t} \vj+\nabla_{x} \rho\right)=-\int_{\mathbb{R}^{3}}\left(\nabla_{x} f+\hat{\vv} \partial_{t} f\right) d \vv \\
%     \vB_{t t}-\Delta \vB=\nabla_{x} \times \vj=\int_{\mathbb{R}^{3}}\left(\hat{\vv} \times \nabla_{x}\right) f d \vv \\
%     f(0, \vx, \vv)=f_{0}(x, v)>0, \vE(0, \vx)=\vE_{0}(\vx), \vB(0, \vx)=\vB_{0}(\vx)
%   \end{array}\right.\end{equation}
%   Then they represent the fields $\vE$ and $\vB$ using the explicit form of the fundamental solution of $\square=\partial_{t}^{2}-\Delta$ in physical space. For example for E they write
%   \begin{equation}
%   \label{eq:rvmEexpression}
%   \vE(t, \vx)=\vE_{0}(t, \vx)+\frac{1}{2 \pi} \int_{0}^{t} d s \int_{C_{t, s}} \frac{d \vy}{|\vy-\vx|}\left(\int\left(\nabla+\hat{\vv} \partial_{t}\right) f(s-|\vy-\vx|, \vy, \vv) d \vv\right)\end{equation}
%     where $\vE_0$ is a solution of the homogeneous equation $\square \vE_0=0$, $C_{t,s}$ is the cone
%     $C_{t,s} = \{|\vy − \vx| \leqslant t − s\}$ and $\nabla= (\partial_1, \partial_2, \partial_3)$. The main trouble is that the existence of $ \nabla+\hat{\vv} \partial_{t}$ requires the estimates for the second order derivatives of $f$ if $||\nabla\vE||_\infty$ estimate requested. Glassey-Strauss eases the requirement by decomposing $ \nabla+\hat{\vv} \partial_{t}$ into fields $T_{i}=\partial_{y_{i}}-\omega_{i} \partial_{t}$, with $\omega_{i}=\frac{(\vy-\vx)_{i}}{|\vy-\vx|}$, and $S=\partial_t + \hat{\vv}\cdot \nabla = -(\vE+\hat{\vv}\times \vB)\cdot \nabla_v$, in terms of which the $\partial_t, \partial_x$ can be reexpressed, \textit{e.g.},
  
%   \begin{equation}\partial_{y_{i}}=\frac{\omega_{i} S}{1+\hat{v} \cdot \omega}+\left(\delta_{i, j}-\frac{\omega_{i} \hat{v}_{j}}{1+\hat{v} \cdot \omega}\right) T_{j}\end{equation}
  
%     Note the $v$ derivative of $S$ operator coorporates with the $v$-integration in equation (\ref{eq:rvmEexpression}). Then the estimate of $||\vE||_\infty$ and $||\vB||_\infty$ by $||f||_\infty$ is acquired, as long as the denominator $1+\hat{\vv}\cdot\omega$ is bounded away from zero guaranted by the assumption (\ref{eq:rvpglassey-straussbound}).
  
%     Furthermore, to estimate the norm of the derivatives of $\vE$ and $\vB$, Glassey-Strauss restricted $\nabla\vE$ and $\nabla\vB$ by $\sup _{\tau \leq t}\|D f(\tau)\|$, with $D = (\nabla x, \nabla v)$, according to the following inequality that for any t in a fixed interval of time $[0,T]$,
  
%     \begin{equation}
%       \left\|\nabla_{x} \vE(t)\right\|_{\infty}+\left\|\nabla_{x} \vB(t)\right\|_{\infty} \leq C_{T}\left(1+\log ^{+}\left(\sup _{\tau \leq t}\|D f(\tau)\|_{\infty}\right)\right).
%     \end{equation}
%     On the other hand, $||Df(t)||_\infty$ estimate is gained by using the transport equation of \eqrvm~ directly.
%   \begin{equation}\|D f(t)\|_{\infty} \leq C_{T} \int_{0}^{t}\left(1+\left\|\nabla_{x} \vE(\tau)\right\|_{\infty}+\left\|\nabla_{x} \vB(\tau)\right\|_{\infty}\right)\|D f(\tau)\|_{\infty} d \tau\end{equation}
  
%   The above two inequalities combined with the Gronwall's inequality gives a bound for all the quantities involved. Continue the proof by a recursive method and then the proof finished.
  
%   \cite*{staffilani2001fourier} reorganized the \eqrvm~ to fit their method based on Fourier Transform and proved the essentially same theorem with \cite*{glassey_symmetric_1985} .
%   \begin{equation}
%     \label{eq:rvmEBmerge}
%     \left\{\begin{array}{l}
%     \partial_{t} f+\hat{\vv} \cdot \nabla_{x} f+ \vect{\alpha}(v) \vect{\Phi} \cdot \nabla_{v} f=0, \quad(\vx, \vv) \in \mathbb{R}^{3} \times \mathbb{R}^{3} \\
%     \vect{\Phi}_{t t}-\Delta \vect{\Phi}=\vect{J}(t, \vx) \\
%     f(0, \vx, \vv)=f_{0}(\vx, \vv)>0, \quad \vect{\Phi}(0, \vx)=\vect{\Phi}_{0}(\vx)
%     \end{array}\right.\end{equation}
  
%     \begin{equation}\begin{aligned}
%       \vect{\Phi}(t, \vx) &=(\vE(t, \vx), \vB(t, \vx)) \\
%       \vect{\alpha}(\vv) \vect{\Phi} &=\vE + \hat{\vv} \times \vB \\
%       \vect{J}(t, \vx) &=\int_{\mathbb{R}^{3}} M(\vv) \nabla_{x} f(t, \vx, \vv) d \vv+\int_{\mathbb{R}^{3}} N(\vv) \vect{\Phi}(t, \vx) \cdot \nabla_{v} f(t, \vx, \vv) d \vv
%       \end{aligned}\end{equation}
%   with $M(\vv)$ and $N(\vv)$ matrices depending only on $\vv$. 
  
%   \begin{theorem}(Klainerman-Staffilani)
%     Consider the IVP (\ref{eq:rvmEBmerge}) with $f_{0} \in C_{0}^{1}\left(\mathbb{R}^{3} \times \mathbb{R}^{3}\right)$ and $\Phi_{0}(x) \in C^{1}\left(\mathbb{R}^{3}\right)$
%   Assume that, on any fixed interval of time $[0, T]$
%   $$
%   \|\Phi\|_{L_{x, t}^{\infty}\left([0, T] \times \mathbb{R}^{3}\right)}<C
%   $$
%   Then the system (\ref{eq:rvmEBmerge}) admits a unique $C^{1}\left([0, T] \times \mathbb{R}^{3}\right)$ solution.
%   \end{theorem}


  

% \subsection{Without Compact Support in "$v$"}

\textbf{在没有初值紧支集的假设下},对 \eqvp 系统的全局适定性问题也有众多的研究者论述。 \eqvp 系统已经成功地解决了对大初值的适定性的问题,
\cite*{pfaffelmoser_global_1992}, \cite*{1991InMat.105..415L} 和 \cite*{schaeffer_global_1991} 等人做了相关工作。
% The question of global well-posedness for \eqvp system without the compact assumpyion in "$v$" has been considered by many authors. The Vlasov-Poisson system has been tackled successfully, for large data, by
% \cite*{pfaffelmoser_global_1992}, \cite*{1991InMat.105..415L}, \cite*{schaeffer_global_1991}.



\cite*{1991InMat.105..415L}, 将\eqvp 将左侧的部分 $\nabla_v f$ 置于右侧视为源项,从而通过特征线的方法导出控制不等式,证明了 $v$ 高阶矩的延续性质。更准确点说便是,如果 $|\vv|^m f_0\in L^1(\bbR^6)\text{ 对任意的 }m<m_0,\text{其中 } m_0> 3$, 则我们有 Vlasov-Poisson 方程的解也满足  $|\vv|^m \ftxv\in L^1(\bbR^6)\text{ 对任意的 }m<m_0,\text{其中 } m_0> 3$ 对任意的 $t>0$。 以高阶矩矩在任意有限时间内的有界性推出了场在任意有限时间内的有界性,从而说明解可以被一直延续下去,也就是说解具有全局存在性。%的延续性为判据也可以说明全局解存在.


% \cite*{1991InMat.105..415L}, based on the representation formula built by the characteristic method considering the source term, proves the propagation of moments in $v$ higher than 3.
% More precisely, if $|\vv|^m f_0\in L^1(\bbR^6)\text{ for all }m<m_0,\text{with } m_0> 3$, then we build a solution
% of Vlasov-Poisson equations satisfying $|\vv|^m \ftxv\in L^1(\bbR^6)\text{ for all }m<m_0,\text{with } m_0> 3$ for any $t>0$. Moreover, for $m_0>6$, Sobolev injections  deduces that $\vE\in L^\infty( [0, T]\times \bbRRR )$ for any $t>0$, and,
% following Horst \footnote{The paper [14] listed in the reference of \cite*{1991InMat.105..415L} is missing, refered as Horst, E.: Global strong solutions of Vlasov's Equation. Necessary and sufficient conditions
% for their existence. Partial differential equations. Banach Cent. Publ. 19, 143 153 (1987)}, $f$ is a smooth function if $f_0(x, v)$ is smooth. 

% For the Vlasov-Maxwell system, \cite*{glassey1986singularity} deduced the result that the classical solution can be globally
% extended with the strong assumption that $f$ has compact support in $\vv$ for all the time. Later ~\cite*{1990MMAS...13..169G} were able to remove the additional support hypothesis for the $2+1/2$ dimensional system whose $\vx \in \bbR^2, \vv \in \bbR^3$. 


% Continuation criterion for the global existence of the Vlasov-Maxwell system
% may be a more relaxed condition than assuming the compact support in "$\vv$". \cite*{robert1989largevelocity} proved that, if the initial data decay at rate $|\vv|^{−7}$ as $|\vv| \to\infty$ and
% the imposed assumption that bound “$\iint_{\bbRRR} |\vv|f(t, \vx, \vv)d\vv ≤ \text{Constant}$” holds, then the solution extends globall.
% Recently, an interesting new continuation criterion was given by \cite*{2016ArRMA.219..445L}, which says that a regular solution can be
% extended as long as k(1+|v|2)θ/2f(t, x, v)kLq xL1 v remains bounded for θ > 2/q, 2 < q ≤ +∞. 

% \cite*{2016ArRMA.219..445L}, \cite*{KUNZE20154413}, \cite*{pallard2015criterion}
% 和 \cite*{PATEL20181841} 近期进一步发展了连续判据。而为了去除对初值紧支集的限制, \cite*{wang_propagation_2018} 研究了 3D \eqrvm 系统关于正则性能否延拓和小初值时系统的长时表现。

% \cite*{2016ArRMA.219..445L}, \cite*{KUNZE20154413}, \cite*{pallard2015criterion}
% and \cite*{PATEL20181841} have developed recent improvements on the continuation criterion. To release the limitation of the compact support and get rid of the assumptions on the solution itself, \cite*{wang_propagation_2018} studied the propagation of regularity and the long time behavior of the 3D RVM system for suitably small initial data.

% Although our main interest is in 3D, we also refer
% readers to [13, 25, 26] for the corresponding results in other dimensions


\section{朗道阻尼}
% \section{Landau Damping}
% \newcommand{\eqlvp}{\hyperref[eq:linearvp]{(lineaized VP)}}
\label{sec:asymptotic}

表示静电相互作用的 Vlasov-Poisson 系统能够刻画等离子体物理领域著名的朗道现象,它是等离子体在大时间尺度时的渐进行为。将 Vlasov-Poisson 方程中的 $\nabla_{v} f$ 替换为 $\nabla_{v}f_0$ 得到的线性化 Vlasov-Poisson 系统 (Linearized Vlasov- Poisson System),便能够一定程度上刻画朗道阻尼的表现;线性朗道阻尼理论便在其之上首先发展起来。

% The electrostatic Vlasov-Poisson equation \eqvp demonstrates the well-known long-time asymptotic phenomenon, named Landau damping in plasma physics. Linearized Vlasov-Poisson equation, the div operator working on $f_0$ rather than $f$, is capable to depict the asymptotic behaviour of the distribution function and linear Landau Damping theory firstly developed.

% in a plasma : given a uniform steady distribution $f_0(\vv)$ and an initial perturbation 
% $f_0(\vx, \vv)$, the equation describes the evolution of this perturbation $f(t, \vx, \vv)$ under the action of the electrostatic potential $\phi(t, \vx)$. 

物理学家首先通过 Fourier-Laplace 变换求解线性化 \eqvp 问题,并且希望确定电势 $\phi(t,\vx)$ 是否在大时间尺度时表示为平面波的形式。他们发现除非存在 $f$ 的 Laplace 变化的解析延拓,除非在涉及到的函数的解析性质足够好的情况下可以做出。然而,这些假设都在许多物理图象中得到了验证。

% Physicists solved the lineaized \eqvp by means of a Fourier-Laplace transform (cf. \cite*{krall_principles_plasma_1973}) and wondered whether the $\phi(t,\vx)$ behaved like plane waves after $t$ being large enough. They found that there is no way to exhibit damped plane waves except by providing an analytical extension of the Laplace transform of $f$, which is only possible with strong analytic assumptions on involved functions. However, these hypotheses are verified in numerous physical situations with these assumptions.

% Physicists solved the lineaized \eqvp by means of a Fourier-Laplace transform (cf. \cite*{krall_principles_plasma_1973}) and wondered whether the $\phi(t,\vx)$ behaved like plane waves after $t$ being large enough. They found that there is no way to exhibit damped plane waves except by providing an analytical extension of the Laplace transform of $f$, which is only possible with strong analytic assumptions on involved functions. However, these hypotheses are verified in numerous physical situations with these assumptions.

除了这种方法之外,相当早期的研究(\cite*{kampen_theory_1955} 和 \cite*{case_plasma_1959})用到了正交模式展开的方法。\cite*{degond_spectral_1986} 研究了线性化 \eqvp 的谱理论来证明它的行为在大时间尺度下表现得如平面波的和。其研究表明要想得到阻尼波展开的分布函数,需要对\eqvp 预解式的解析延拓。

% Beside this approach, another successful theory which has been developed long time ago is established by \cite*{kampen_theory_1955} and \cite*{case_plasma_1959}, using a 
% "normal mode expansion".  
% \cite*{degond_spectral_1986} studied the spectral theory of the linearized \eqvp in order to prove that its solution behaves, for large times, like a sum of plane waves. It is shown that to obtain the distribution function expansion expressed by damped waves, an analytical extension of the resolvent of the \eqvp  is necessary. 

线性的研究理论是朗道阻尼研究中很长一段时间的焦点,而在此之上的非线性朗道阻尼理论,则由 \cite*{mouhot2011} 在近期给出。其阻尼线性被重新诠释为一种正则性在动力学量和空间相关变量之间的转移,而不是能量的交换。这项研究还揭示了阻尼的驱动机制确实是相混合(phase mixing)。
% Beyond the linearized study, which has been studied for a long period in the theory of Landau damping. \cite*{mouhot2011} established the theory of exponential Landau damping in analytic regularity where the damping phenomenon is reinterpreted in terms of transfer of regularity between kinetic and spatial variables, rather than exchanges of energy. The study revealed that the phase mixing is the driving mechanism of damping.

% Wenyin is unfamiliar with the topic of operator theory employed in the relevant papers, so he might just give a short introduction about this topic. However, this topic is tightly related to the Landau damping, Wenyin would try to figure it out though he has little idea about \emph{the resolvent, Dunfold formula and the spectrum of an operator.}

% \begin{equation}
% \label{eq:linearvp}
% \text{(linearized VP)}\quad
% \frac{\partial f}{\partial t}+\vv \cdot \nabla_x f+\nabla_x \phi  \cdot \nabla_v f_{0}(\vv)=0 \end{equation}

% \eqlvp~ is transformed to such a form $\dot{f}=T \cdot f ; \quad f(0)=f_0$ that could be solved by semigroup theory: $f(t) = \exp(tT) \cdot f_0$. The Dunford formula \inlinecite{hille_functional_1948} then 
%   relates it to the resolvent $R_\lambda = (\lambda - T)^{-1}$. A deformation of the path of integration 
%   in the Dunford formula allows to give an asymptotic expansion for $f$ without damped wave.


% \begin{equation}f(x, v, t)=\sum_{s=1}^{S} \alpha_{s}(v) e^{\lambda_{s} t+i n_{s} x}+\mathcal{O}\left(e^{r t}\right), \quad r<\min_{1 \leqslant s \leqslant S}\left(\operatorname{Re} \lambda_{s}\right)\end{equation}
%   where $r$ can be negative,  and $\alpha_s(v)$ are well-defined functions of $v$. Then the poles $\lambda_s$ of this extension are no longer eigenvalues of $T$ and must be interpreted as eigenmodes, associated to "generalized eigenfunctions" which actually are linear functionals on a Banach space of 
% analytic functions. 

% This work is an attempt at a mathematical explanation of Landau 
% damping in terms of eigenmodes and scattering theory.
% It is shown that the potential $\phi(t, x)$ and the transform $f(t, x, \xi)$ of $f(t, x, v)$ actually admits the expansions:

% \begin{equation}\phi(x, t)=\sum_{s=1}^{S} c_{s} e^{\lambda_{s} t+i n_{s} x}+\mathcal{O}\left(e^{r t}\right), \quad r< \min_{1 \leqslant s \leqslant S}\left(\operatorname{Re} \lambda_{s}\right)\end{equation}
% \begin{equation}\check{f}(x, \xi, t)=\sum_{s=1}^{S} \breve{\alpha}_{s}(\xi) e^{\lambda_{s} t+i n_{s} x}+\mathcal{O}\left(e^{r t} \right), \quad r<\min_{1 \leqslant s \leqslant S}\left(\operatorname{Re} \lambda_{s}\right)   \end{equation}
  
  % \section{Stability}
  % \label{sec:stability}
  % Large amount of attention has been put on the problem of stability of stationary solutions of the Vlasov-Poisson system, both in the stellar dynamics and the plasma physics cases. The energy-Casimir 
  % method has been used to prove non-linear stability for various conservative systems, and \cite*{rein_non-linear_1994} employed the method to prove non-linear stability of the Vlasov-Poisson system in three grometrically different setting. The three settings are the situations where the ion density is replaced by a given fixed ion background, the plasma can be spatially periodic, or can be restricted to a bounded domain. 
  
  
  % With the exception of the first case, stationary solutions exist in these settings and also in the stellar dynamic case. 
  % In the physics literature there are numerous investigations of the problem of stability of these stationary solutions, both linear and non-linear, and we refer to [1,2]\footnote{Not yet found} and the monographs [4,5]\footnote{Not yet found} for references. In contrast, very little rigorous mathematics exists on this problem. In [2,9]\footnote{Not yet found} non-linear stability of stationary solutions in a spatially periodic, plasma physics setting is established for the Vlasov-Poisson system and the 
  % relativistic Vlasov-Maxwell system, respectively, see also [lo, 111\footnote{Not yet found}. The problem of linear stability is investigated in [l]\footnote{Not yet found}, both for the plasma physics and the stellar dynamics cases. The phenomenon of Landau damping is established mathematically 
  % in [ 6 ]\footnote{Not yet found} for the one-dimensional, linearized Vlasov-Poisson system. 
  % The starting point of the present investigation is a general method to assert 
  % non-linear stability of stationary solutions for (infinite-dimensional, degenerate) 
  % Hamiltonian systems, which is presented in [S]\footnote{Not yet found}. We briefly review this method. Let the 
  % system under consideration be described by the equation of motion 
  
  % $$\dot{u} = A(u)$$
  
  % on some state space $X , A : D(A) \rightarrow X $ a (non-linear) operator, and let $u_0$ be the 
  % stationary solution whose stability we want to investigate. The following steps lead to 
  % a stability result for $u_0$:
  
  % \begin{enumerate}
  %   \item Find the energy (Hamiltonian) $H :X + \bbR$ of the system; $d H ( u ( t ) )/dt = 0$ along 
  %   solutions. 
  %   \item Relate $u_0$ to another conserved quantity Casimir functional $C : X + \bbR$ such that $u_0$ is a critical point of $H_C := H + C$, i.e. $DH_C(u_0) = 0$. 
  %   \item Show that the quadratic part in the expansion of $H_c$ at $u_0$ 
  %   $$H_C(u) = H_C(u_00) + DH_C(u_0)(u - u_0) + D^2H_C(u_0)(u - u_0, u - u_0) + ... $$
  %   is positive definite, more precisely, find a norm $|| \cdot ||_a $ on X such that 
  %   $$H_C(u) - H_C(u_0) - DH_C(u_0)(U - u_0) \geq  C||u-u_0||^2_a \in X,$$ 
  %   for some $C > 0$. 
  %   \item Find a norm $||\cdot ||_b$ on $X$ with respect to which $H_c$ is continuous at $u_0$. 
  % \end{enumerate}
  
  
  % If Steps (1)-(3) can be carried through, then for any solution 
  
  % and with Step (4) we conclude that for any $\varepsilon > 0$ there exists $\delta > 0$ such that $||u(0)-u_0||_b < \delta$ implies  $$\left\|u(t)-u_{0}\right\|^{2}_a \leqslant \frac{1}{C}\left|H_{C}(u(0))-H_{C}\left(u_{0}\right)\right| ,$$ \textit{i.e}. $u_0$ is non-linearly stable. 
  
  % Though the energy-Casimir method is elegant and appealing, in [S] 
  % it is pointed out that the appearance of large velocities could cause the method to run into trouble.
  
\section{符号标记}

文献综述时描述局部解的适定性问题主要参考 \cite*{HorstClasssicalI},大部分标记和定义沿用了其原文,一定程度上做了修改。

文章中的 $C$ 通常表示不依赖于初始条件的常数,而 $K$ 则是依赖于初始条件的常数,它们在我们研究给定 Cauchy 初值条件的时候都可视为常数。函数符号方面则用 $C_+(I):=\{f:I\rightarrow [0,\infty)\bigg| f \text{ 连续且单调增}\}$ 表示我们用来控制的函数空间,其中 $I$ 是一个区间,$H$ 常常表示一个 $C_{+}(I)$ 或 $C_{+}([0,\infty))$ 集合中的函数,而用 $h$ 表示通过在一段时间对某个量,如电场 $\vE$ 的大小,取上确界 $\sup$ 得到的单调增函数。通常当需要对某个量进行控制的时候,我们会取该量在一段时间上的上确界作为新的函数,并通过 $C_+(I)$ 中的函数对它进行控制。

函数积分的时候进行积分域的分割常将不同的积分项命名为 $I_1$, $I_2$ 等,由于只是局部的使用,为简介起见,在不同的积分式分割中积分项均以下标 $1$ 开始,应不会产生混淆。

% We also use constants $C$ which do not depend on $f_0$ and $K$ which do. The first index $i$ always denotes the number of the lemma or theorem, where it is defined.

% For any two numbers $A$ and $B$, we use $A \lesssim B$ and $B \gtrsim A$ to denote $A \lesssim C B$, where $C$ is an absolute constant.



$\omega_{N}:=2 \cdot \pi^{N / 2} / \Gamma(N / 2)$ 是 $N$ 维空间中 $(N-1)$ 维的单位球面的表面积。


% $\omega_{N}:=2 \cdot \pi^{N / 2} / \Gamma(N / 2)$ is the surface area of the $(N-1)$ -dimensional sphere. If $I \subset\left[0, \infty\right)\text { we let } C_{+}(I):=\{f: I \rightarrow[0, \infty)| f\text { is continuous and non-decreasing } $ on $I\}$

本文中谈到的积分和测度总是基于 Lebesgue 意义下的,$\mathrm{L}_{\infty}\left(\bbR^{M}, \bbR^{L}\right)$ 和 $\|\cdot\|_{\infty}$ 遵循实分析的通常定义。 $\mathrm{L}_{p}\left(\bbR^{M}\right):=L_{p}\left(\bbR^{M}, \bbR\right)$
% Integration and measurability are always meant with respect to the Lebesgue measure. $\mathrm{L}_{\infty}\left(\bbR^{M}, \bbR^{L}\right)$ and $\|\cdot\|_{\infty}$ are also defined as usual. $\mathrm{L}_{p}\left(\bbR^{M}\right):=L_{p}\left(\bbR^{M}, \bbR\right)$

对所有的函数 $f: \bbR^{M} \rightarrow \bbR^{L}$, Lipschitz 常数按惯例为 $\operatorname{lip}(f):=\sup _{z \neq w}|z-w|^{-1}|f(z)-f(w)|$。
$\operatorname{Lip}\left(\bbR^{M}, \bbR^{L}\right):=\left\{f: \bbR^{M} \rightarrow \bbR^{L} | \operatorname{lip}(f)<\infty\right\}$,  $\operatorname{Lip}\left(\bbR^{M}\right):=\operatorname{Lip}\left(\bbR^{M}, \bbR\right)$
% For all $f: \bbR^{M} \rightarrow \bbR^{L}$ we let $\operatorname{lip}(f):=\sup _{z \neq w}|z-w|^{-1}|f(z)-f(w)|$



% 在偏微分方程中各种量互相控制时,有时常数在不等式中并不特别重要,因此我们还用 $A \lesssim B$ 和 $B \gtrsim A$ 这样的符号来表示  $A \lesssim C B$,其中 $C$ 可以是依赖于初值条件的常数。





















% \section{定理环境}
% \label{sec:theorem}

% 给大家演示一下各种和证明有关的环境:

% \begin{assumption}
% 待月西厢下,迎风户半开;隔墙花影动,疑是玉人来。
% \begin{align}
%   \label{eq:eqnxmp}
%   c & = a^2 - b^2 \\
%     & = (a+b)(a-b)
% \end{align}
% \end{assumption}

% 千辛万苦,历尽艰难,得有今日。然相从数千里,未曾哀戚。今将渡江,方图百年欢笑,如
% 何反起悲伤?(引自《杜十娘怒沉百宝箱》)

% \begin{definition}
% 子曰:「道千乘之国,敬事而信,节用而爱人,使民以时。」
% \end{definition}


% \begin{proposition}
%  曾子曰:「吾日三省吾身 —— 为人谋而不忠乎?与朋友交而不信乎?传不习乎?」
% \end{proposition}


% \begin{remark}
% 天不言自高,水不言自流。
% \begin{gather*}
% \begin{split}
% f_0(x,z)
% &=z-\gamma_{10}x-\gamma_{mn}x^mz^n\\
% &=z-Mr^{-1}x-Mr^{-(m+n)}x^mz^n
% \end{split}\\[6pt]
% \begin{align} \zeta^0&=(\xi^0)^2,\\
% \zeta^1 &=\xi^0\xi^1,\\
% \zeta^2 &=(\xi^1)^2,
% \end{align}
% \end{gather*}
% \end{remark}


% \begin{axiom}
% 两点间直线段距离最短。
% \begin{align}
% x&\equiv y+1\pmod{m^2}\\
% x&\equiv y+1\mod{m^2}\\
% x&\equiv y+1\pod{m^2}
% \end{align}
% \end{axiom}

% 《彖曰》:大哉乾元,万物资始,乃统天。云行雨施,品物流形。大明始终,六位时成,时
% 乘六龙以御天。乾道变化,各正性命,保合大和,乃利贞。首出庶物,万国咸宁。

% 《象曰》:天行健,君子以自强不息。潜龙勿用,阳在下也。见龙再田,德施普也。终日乾
% 乾,反复道也。或跃在渊,进无咎也。飞龙在天,大人造也。亢龙有悔,盈不可久也。用九,
% 天德不可为首也。   

% \newcommand\dif{\mathop{}\!\mathrm{d}}
% \begin{lemma}
% 《猫和老鼠》是我最爱看的动画片。
% \begin{multline*}%\tag*{[a]} % 这个不出现在索引中
% \int_a^b\biggl\{\int_a^b[f(x)^2g(y)^2+f(y)^2g(x)^2]
%  -2f(x)g(x)f(y)g(y)\dif x\biggr\}\dif y \\
%  =\int_a^b\biggl\{g(y)^2\int_a^bf^2+f(y)^2
%   \int_a^b g^2-2f(y)g(y)\int_a^b fg\biggr\}\dif y
% \end{multline*}
% \end{lemma}

% 行行重行行,与君生别离。相去万余里,各在天一涯。道路阻且长,会面安可知。胡马依北
% 风,越鸟巢南枝。相去日已远,衣带日已缓。浮云蔽白日,游子不顾返。思君令人老,岁月
% 忽已晚。  弃捐勿复道,努力加餐饭。

% \begin{theorem}\label{the:theorem1}
% 犯我强汉者,虽远必诛\hfill —— 陈汤(汉)
% \end{theorem}
% \begin{subequations}
% \begin{align}
% y & = 1 \\
% y & = 0
% \end{align}
% \end{subequations}
% 道可道,非常道。名可名,非常名。无名天地之始;有名万物之母。故常无,欲以观其妙;
% 常有,欲以观其徼。此两者,同出而异名,同谓之玄。玄之又玄,众妙之门。上善若水。水
% 善利万物而不争,处众人之所恶,故几于道。曲则全,枉则直,洼则盈,敝则新,少则多,
% 多则惑。人法地,地法天,天法道,道法自然。知人者智,自知者明。胜人者有力,自胜
% 者强。知足者富。强行者有志。不失其所者久。死而不亡者寿。

% \begin{proof}
% 燕赵古称多感慨悲歌之士。董生举进士,连不得志于有司,怀抱利器,郁郁适兹土,吾
% 知其必有合也。董生勉乎哉?

% 夫以子之不遇时,苟慕义强仁者,皆爱惜焉,矧燕、赵之士出乎其性者哉!然吾尝闻
% 风俗与化移易,吾恶知其今不异于古所云邪?聊以吾子之行卜之也。董生勉乎哉?

% 吾因子有所感矣。为我吊望诸君之墓,而观于其市,复有昔时屠狗者乎?为我谢
% 曰:“明天子在上,可以出而仕矣!” \hfill —— 韩愈《送董邵南序》
% \end{proof}

% \begin{corollary}
%   四川话配音的《猫和老鼠》是世界上最好看最好听最有趣的动画片。
% \begin{alignat}{3}
% V_i & =v_i - q_i v_j, & \qquad X_i & = x_i - q_i x_j,
%  & \qquad U_i & = u_i,
%  \qquad \text{for $i\ne j$;}\label{eq:B}\\
% V_j & = v_j, & \qquad X_j & = x_j,
%   & \qquad U_j & u_j + \sum_{i\ne j} q_i u_i.
% \end{alignat}
% \end{corollary}

% 迢迢牵牛星,皎皎河汉女。
% 纤纤擢素手,札札弄机杼。
% 终日不成章,泣涕零如雨。
% 河汉清且浅,相去复几许。
% 盈盈一水间,脉脉不得语。

% \begin{example}
%   大家来看这个例子。
% \begin{equation}
%   \label{ktc}
%   \begin{cases}
%     \nabla f(\bm{x}^*) - \sum_{j=1}^p \lambda_j \nabla g_j(\bm{x}^*)
%       = 0 \\[0.3cm]
%     \lambda_j g_j(\bm{x}^*) = 0, \quad j = 1, 2, \dots, p \\[0.2cm]
%     \lambda_j \ge 0, \quad j = 1, 2, \dots, p.
%   \end{cases}
% \end{equation}
% \end{example}

% \begin{exercise}
%   请列出 Andrew S. Tanenbaum 和 W. Richard Stevens 的所有著作。
% \end{exercise}

% \begin{conjecture} \textit{Poincare Conjecture} If in a closed three-dimensional
%   space, any closed curves can shrink to a point continuously, this space can be
%   deformed to a sphere.
% \end{conjecture}

% \begin{problem}
%  回答还是不回答,是个问题。
% \end{problem}

% 如何引用定理~\ref{the:theorem1} 呢?加上 \cs{label} 使用 \cs{ref} 即可。妾发
% 初覆额,折花门前剧。郎骑竹马来,绕床弄青梅。同居长干里,两小无嫌猜。 十四为君妇,
% 羞颜未尝开。低头向暗壁,千唤不一回。十五始展眉,愿同尘与灰。常存抱柱信,岂上望夫
% 台。 十六君远行,瞿塘滟滪堆。五月不可触,猿声天上哀。门前迟行迹,一一生绿苔。苔深
% 不能扫,落叶秋风早。八月蝴蝶来,双飞西园草。感此伤妾心,坐愁红颜老。

% \section{参考文献}
% \label{sec:bib}
% 当然参考文献可以直接写 \cs{bibitem},虽然费点功夫,但是好控制,各种格式可以自己随意改
% 写。

% 本模板推荐使用 BIB\TeX,分别提供数字引用(\texttt{thuthesis-numeric.bst})和作
% 者年份引用(\texttt{thuthesis-author-year.bst})样式,基本符合学校的参考文献格式
% (如专利等引用未加详细测试)。看看这个例子,关于书的~\cite*{tex, companion,
%   ColdSources},还有这些~\cite*{Krasnogor2004e, clzs, zjsw},关于杂志
% 的~\cite*{ELIDRISSI94, MELLINGER96, SHELL02},硕士论文~\cite*{zhubajie,
%   metamori2004},博士论文~\cite*{shaheshang, FistSystem01},标准文
% 件~\cite*{IEEE-1363},会议论文~\cite*{DPMG,kocher99},技术报告~\cite*{NPB2},电子文
% 献~\cite*{chuban2001,oclc2000}。
% 若使用著者-出版年制,中文参考文献~\cite*{cnarticle}应增加
% \texttt{key=\{pinyin\}} 字段,以便正确进行排序~\cite*{cnproceed}。
% 另外,如果对参考文献有不如意的地方,请手动修改 \texttt{bbl} 文件。

% 有时候不想要上标,那么可以这样~\inlinecite{shaheshang},这个非常重要。

% 有时候一些参考文献没有纸质出处,需要标注 URL。缺省情况下,URL 不会在连字符处断行,
% 这可能使得用连字符代替空格的网址分行很难看。如果需要,可以将模板类文件中
% \begin{verbatim}
% \RequirePackage{hyperref}
% \end{verbatim}
% 一行改为:
% \begin{verbatim}
% \PassOptionsToPackage{hyphens}{url}
% \RequirePackage{hyperref}
% \end{verbatim}
% 使得连字符处可以断行。更多设置可以参考 \texttt{url} 宏包文档。

% \section{公式}
% \label{sec:equation}
% \renewcommand\vec{\symbf}
% \newcommand\mat{\symbf}
% 贝叶斯公式如式~(\ref{equ:chap1:bayes}),其中 $p(y|\vec{x})$ 为后验;
% $p(\vec{x})$ 为先验;分母 $p(\vec{x})$ 为归一化因子。
% \begin{equation}
% \label{equ:chap1:bayes}
% p(y|\vec{x}) = \frac{p(\vec{x},y)}{p(\vec{x})}=
% \frac{p(\vec{x}|y)p(y)}{p(\vec{x})}
% \end{equation}

% 论文里面公式越多,\TeX{} 就越 happy。再看一个 \pkg{amsmath} 的例子:
% \newcommand{\envert}[1]{\left\lvert#1\right\rvert}
% \begin{equation}\label{detK2}
% \det\mat{K}(t=1,t_1,\dots,t_n)=\sum_{I\in\vec{n}}(-1)^{\envert{I}}
% \prod_{i\in I}t_i\prod_{j\in I}(D_j+\lambda_jt_j)\det\vec{A}
% ^{(\lambda)}(\overline{I}|\overline{I})=0.
% \end{equation}

% 前面定理示例部分列举了很多公式环境,可以说把常见的情况都覆盖了,大家在写公式的时
% 候一定要好好看 \pkg{amsmath} 的文档,并参考模板中的用法:
% \begin{multline*}%\tag{[b]} % 这个出现在索引中的
% \int_a^b\biggl\{\int_a^b[f(x)^2g(y)^2+f(y)^2g(x)^2]
%  -2f(x)g(x)f(y)g(y)\dif x\biggr\}\dif y \\
%  =\int_a^b\biggl\{g(y)^2\int_a^bf^2+f(y)^2
%   \int_a^b g^2-2f(y)g(y)\int_a^b fg\biggr\}\dif y
% \end{multline*}

% 其实还可以看看这个多级规划:
% \begin{equation}
%   \label{bilevel}
%   \begin{cases}
%     \max_{\bm{x}} F(\bm{x}, y_1^*, y_2^*, \dots, y_m^*) \\
%       \text{subject to:} \\
%       \qquad G(\bm{x}) \le 0 \\
%       \qquad (y_1^*, y_2^*, \dots, y_m^*) \text{ solves problems }
%         (i = 1, 2, \dots, m) \\
%       \qquad
%         \begin{cases}
%           \max_{\bm{x}} f_i(\bm{x}, y_1, y_2, \dots, y_m) \\
%           \text{subject to:} \\
%           \qquad g_i(\bm{x}, y_1, y_2, \dots, y_m) \le 0.
%         \end{cases}
%   \end{cases}
% \end{equation}
% 这些跟规划相关的公式都来自于刘宝碇老师《不确定规划》的课件。
